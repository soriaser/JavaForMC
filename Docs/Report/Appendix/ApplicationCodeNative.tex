\chapter{Native Application Code}\label{Appx:AppCodeNative}
Following C code shows the native application code used for direct interface circuit fo resistor sensors:
\medskip
\begin{lstlisting}[
caption={Native C application code},
captionpos=b,
label={cod:Appx:AppCodeNative:NativeCode},
language=C]
/*
 * File:   main.c
 * Author: Sergio Soria
 *
 * Created on 25 de enero de 2014, 17:05
 */

#include <xc.h>
#include <stdio.h>
#include "configuration.h"

#pragma config OSC      = HS
#pragma config IESO     = OFF
#pragma config FCMEN    = OFF
#pragma config WDT      = OFF
#pragma config BOREN    = OFF
#pragma config PWRT     = OFF
#pragma config MCLRE    = ON
#pragma config PBADEN   = OFF
#pragma config LVP      = OFF
#pragma config CPD      = OFF
#pragma config CP0      = OFF
#pragma config CP1      = OFF
#pragma config CP2      = OFF
#pragma config CP3      = OFF

#define _XTAL_FREQ 4000000
#define BAUDRATE 2400

// Booleans true and false values
#define TRUE    1
#define FALSE   0

// Byte value required to be received in order to start measures
#define RX_BYTE 0x01

// Application states
#define ST_MEASURE_NX   0
#define ST_MEASURE_NC1  1
#define ST_MEASURE_NC2  2

unsigned char TRISBValue = 0x00;
// Temporal array containing times measured
unsigned short g_Times[3];
// Temporal variable containing time measured
unsigned short g_Time = 0;

// If TRUE, measurements start. If FALSE, nothing happens
int g_Start     = FALSE;
// Current application state. Initially is measuring Nx
int g_State     = ST_MEASURE_NX;
// If TRUE, next time measurement can be done. If FALSE, not.
int g_Next      = FALSE;

void interrupt highISR(void)
{
#if (TIMER == TIMER1)
    // Stop Timer
    TMR1ON = 0;
    // Get Time
#ifdef CCP1
    g_Time = CCPR1;
#else // CCP1
    g_Time = READTIMER1();
#endif // (TIMER == TIMER1)

#elif (TIMER == TIMER2)
    // Stop Timer
    TMR2ON = 0;
    // Get Time
    g_Time = TMR2;
#endif // (TIMER == TIMER2)
    // Enable next time measurement
    g_Next = TRUE;

    // Clear interrupt
#ifdef CCP1
    CCP1IF = 0;
#else
    INT0IF = 0;
#endif // CCP1
}


void interrupt low_priority lowISR(void)
{
    if ((RCIF) && ((RCREG & RX_BYTE) == RX_BYTE)) {
        // Clear interrupt
        RCIF = 0;
        // If over run error, then reset the receiver
        if(OERR) {
            CREN = 0;
            CREN = 1;
        }
        // Indicates that measure can start
        g_Start = TRUE;
    }
}

#ifdef _STDIO_H_
void putch(char data)
{
    // Wait for previous transmission to finish.
    while (!TXIF) {
        continue;
    }

    // Write byte to send.
    TXREG = data;
}
#endif // _STDIO_H_

void main(void) {
    // Force to discharge capacitor through auxiliar Pin 5
    TRISB = 0x0F;
    PORTB = 0x00;

    __delay_ms(30);

    // Configure interrupts
    // --------------------
    //
    // Enable high and low priority interrupts int INT0 interrupt
    INTCON  = 0xD0;

#ifdef CCP1
    // CCP1 Interrupt enable bit
    CCP1IE = 1;
    // CCP1 interrupt as high priority
    CCP1IP = 1;
#else
    // Enable PORTB pull-ups and let RB0 interrupt on falling edge
    INTCON2 = 0x80;
#endif // CCP1

#ifdef CCP1
    CCP1CON = 0x05;
#else
    // If external Schmitt trigger is used in RB0, the edge interruption
    // should be swtiched to interrupt on rising edge
    INTEDG0 = 1;
#endif // CCP1

    // Enable interrupt priority
    IPEN = 1;

    // Configure Serial Port
    // ---------------------
    //
    // - RC6 = Asyncronous serial data ouput.
    // - RC7 = Syncronous serial receive data input.
    TRISC6 = 1;
    TRISC7 = 1;
    // - 8-bit Baud Rate
    BRG16 = 0;
    BRGH  = 0;
    SPBRG = ((_XTAL_FREQ / 64) / BAUDRATE) - 1;
    // - 8-bit transmission and reception
    TX9   = 0;
    // - Asynchronous mode
    SYNC  = 0;
    RX9   = 0;
    // Enable Serial port and receive mode
    SPEN  = 1;
    SREN  = 0;
    CREN  = 1;
    // Enable Rx interrupt and disable Tx interrupt
    RCIE  = 1;
    TXIE  = 0;

    // Set Rx interrupt as low level
    RCIP = 0;

    // Reset and enable transmitter
    TXEN  = 0;
    TXEN  = 1;

#ifdef CCP1
    // Configure CCP Timer
    // -------------------
    // - Timer 1 as CCP clock
    T3CON = 0x08;
    // RC2 as input
    TRISC2 = 1;
#endif //CCP1

#if (TIMER == TIMER1)
    // Configure Timer 1.
    // ------------------
    // - 16-bits
    // - 1:1 Prescale
    // - Internal clock
    //
    // Disables Timer1 overflow interrupt
    TMR1IE = 0;

    T1CON = 0x88;
    // Initially Timer 1 OFF
    TMR1ON = 0;

    // Clear counter values
    TMR1L = 0x00;
    TMR1H = 0x00;
#elif (TIMER == TIMER2)
    // Configure Timer 2.
    // ------------------
    // - 1:1 Postcale
    // - 1:16 Prescale
    //
    T2CON = 0x02;
    // Initially Timer 2 OFF
    TMR2ON = 0;
    // Disable the TMR2 to PR2 match interrupt
    TMR2IE = 0;
    // Clear counter values
    TMR2 = 0x00;
#endif // (TIMER == TIMER1)

    while (TRUE) {
        // Wait until receive byte to start
        while (g_Start) {
            // Charge capacitor throug auxiliar Pin 5
            TRISB = 0x0F;
            PORTB = 0x10;

            __delay_ms(30);

            // Next step is not goig to be executed until RB0
            // interrupt finishes
            g_Next = FALSE;

#if (TIMER == TIMER1)
            // Start Timer1
            WRITETIMER1(0x0000);
#elif (TIMER == TIMER2)
            // Start Timer 2
            TMR2 = 0x00;
            TMR2ON = 1;
#endif // TIMER

#ifdef CCP1
            CCPR1 = (unsigned short) 0;
#endif // CCP1

            switch (g_State) {
                case ST_MEASURE_NX:
                    // Discharge through Pin 2
                    TRISBValue = 0x1D;
                    break;
                case ST_MEASURE_NC1:
                    // Discharge through Pin 3
                    TRISBValue = 0x1B;
                    break;
                case ST_MEASURE_NC2:
                    // It is going to be the last measure
                    g_Start = FALSE;
                    // Discharge through Pin 4
                    TRISBValue = 0x17;
                    break;
            }

            TRISB = TRISBValue;

#if (TIMER == TIMER1)
            TMR1ON = 1;
#elif (TIMER == TIMER2)
            TMR2ON = 1;
#endif // (TIMER == TIMER1)

            // All values as logical '0'
            PORTB = 0x00;

            asm("L1: BTFSS _g_Next,0");
            asm("GOTO L1");

            // Store time
            g_Times[g_State] = g_Time;
            // Next step
            g_State++;

            // brief delay
            __delay_ms(20);
        }

        if (g_State > ST_MEASURE_NC2) {
            // Send values
            printf("%d;", g_Times[ST_MEASURE_NX]);
            printf("%d;", g_Times[ST_MEASURE_NC1]);
            printf("%d", g_Times[ST_MEASURE_NC2]);
            printf("\n");

            // Restart state
            g_State = ST_MEASURE_NX;
        }
    }
}

\end{lstlisting}
\medskip

