%%%%%%%%%%%%%%%%%%%%%%%%%%%%%%%%%%%%%%%%%%%%%%%%%%%%%%%%%%%%%%%%%%%%%%%%%%%%%
%%%%%%                                                                  %%%%% 
%%%%%%       Fitxer de dades per la memoria TFC/PFC de l'EETAC          %%%%% 
%%%%%%                                                                  %%%%% 
%%%%%%%%%%%%%%%%%%%%%%%%%%%%%%%%%%%%%%%%%%%%%%%%%%%%%%%%%%%%%%%%%%%%%%%%%%%%%
%%%%%%%%%%%%%%%%%%%%%%%%%%%%%%%%%%%%%%%%%%%%%%%%%%%%%%%%%%%%%%%%%%%%%%%%%%%%%
%%                                                                         %%
%%          Autor: Xavier Prats i Menendez (xavier.prats@upc.edu)          %% 
%%                  Technical University of Catalonia (UPC)                %%
%%                                                                         %%
%%%%%%%%%%%%%%%%%%%%%%%%%%%%%%%%%%%%%%%%%%%%%%%%%%%%%%%%%%%%%%%%%%%%%%%%%%%%%
%%      This work is licensed under the Creative Commons  Attribution-     %%
%%   -Noncommercial-Share Alike 3.0 Spain License. To view a copy of this  %% 
%%    license, visit http://creativecommons.org/licenses/by-nc-sa/3.0/es/  %%
%%    or send a letter to Creative Commons, 171 Second Street, Suite 300,  %%
%%                  San Francisco,California, 94105, USA.                  %%
%%%%%%%%%%%%%%%%%%%%%%%%%%%%%%%%%%%%%%%%%%%%%%%%%%%%%%%%%%%%%%%%%%%%%%%%%%%%%
%% Versio 2.1 - Juliol 2012                                                %%
%%%%%%%%%%%%%%%%%%%%%%%%%%%%%%%%%%%%%%%%%%%%%%%%%%%%%%%%%%%%%%%%%%%%%%%%%%%%%

%%%%%%%%%%%%%%%%%%%%%%%%%%%%%%%%%%%%%%%%%%%%%%%%%%%%%%%%%%%%%%%%%%%%%%%%%%%%%%%
%%  VARIABLES A CONFIGURAR                                                  %%%
%%%%%%%%%%%%%%%%%%%%%%%%%%%%%%%%%%%%%%%%%%%%%%%%%%%%%%%%%%%%%%%%%%%%%%%%%%%%%%%

%% - Projecte o Treball de Fi de Carrera?
%%      PFC = true   -> Projecte de Fi de Carrera
%%      PFC = false  -> Treball  de Fi de Carrera
\setboolean{PFC}{false}

%% - Escollir la titulació
%\titulacio{Enginyeria Tècnica Aeronàutica, especialitat Aeronavegació}
%\titulacio{Enginyeria T\`ecnica de Telecomunicaci\'o, especialitat Sistemes de Telecomunicaci\'o}
%\titulacio{Enginyeria T\`ecnica de Telecomunicaci\'o, especialitat Telem\`atica}
%\titulacio{Enginyeria de Telecomunicaci\'o (segon cicle)}
% Modificació respecte a la versió 2.1 - Iván Padilla Montero - Juliol 2014
%\titulacio{Grau en Enginyeria d'Aeronavegaci\'o}
%\titulacio{Grau en Enginyeria d'Aeroports}
%\titulacio{Grau en Enginyeria Telemàtica}
%\titulacio{Grau en Enginyeria de Sistemes de Telecomunicació}
\titulacio{MASTEAM}


%% - Configurar els idiomes del document
%% Si l'idioma PRINCIPAL del document es l'angles, posar aquesta variable a true
\setboolean{Leng}{true}

%% Escollir entre catala i castella (idioma principial, o nomes pel resum en cas que l'idioma principal sigui anglès)
%%  catala = true   -> idioma principal (o només resum) en Català
%%  catala = false  -> idioma principal (o només resum) en Castella
\setboolean{Lcat}{false}

%% Titol del document en l'idioma principal del document 
\titol{Development of a JVM for low and mid range microcontrollers and comparison with native applications.}

%% Titol del document en anglès (Per l'apartat overview)
\titolE{Development of a JVM for low and mid range microcontrollers and comparison with native applications.}

%% Titol del document en catala/castella (Per l'apartat resum)
\titolC{Desarrollo de una JVM para microcontroladores de baja y media gama, y comparación con applicaiones nativas}


%% - Nombre d'autors del TFC/PFC?
%%      UNautor = true   Un sol autor
%%      UNautor = false  Més d'un autor
\setboolean{UNautor}{true}

%% - Nom del(s) Autor(s) del document
%% NOTA: En cas de mes d'un autor cal posar la comana \and entre els
%%        noms dels autors
\autor{Sergio Soria Nieto}

%% - Nombre de directors del TFC/PFC. Tipicament 1 o 2
%%      UNdirector = true   Un sol director
%%      UNdirector = false  Dos directors
\setboolean{UNdirector}{true}

%% - Nom del Director del TFC/PFC
\director{Josep Jordana}

%% - Nom del segon director en cas de tenir-lo:
\segonDirector{Nom2 Cognoms2}


%% - Es vol incloure una dedicatoria?
%%      dedicatoria = true   -> S'afegeix una pagina amb \textDedicatoria
%%      dedicatoria = true   -> No s'afegeix dedicatoria
%% NOTA: no confondre dedicatòria amb agraïments. Una dedicatoria sol ser
%%       un missatge curt d'una o dues frases màxim a la persona, o persones
%%       a les quals es dedica el treball. 
%%       Els agraïments poden ser extensos i l'autor pot agraïr a diverses
%%       persones coses diferents en funció de l'ajuda rebuda, per exemple. 
%%       Si es volen incloure agraïments, fer-ho al fitxer de la 
%%       memòria creant una secció nova amb  \chapter*{Agraïments}
\setboolean{dedicatoria}{true}
\textDedicatoria{Escriure aqui \\ la dedicatòria}

%% - Es vol incloure una pagina d'index de figures?
\setboolean{paginaLOF}{true}  % List of Figures

%% - Es vol incloure una pagina d'index de taules?
\setboolean{paginaLOT}{true}  % List of Tables 

%% - El projecte ha estat supervisat per alguna persona externa? 
%%   (NOMES en cas de practiques en empresa)
%%      supervisor = true    -> Hi ha un supervisor
%%      supervisor = false   -> No hi ha un supervisor
\setboolean{supervisor}{true}

%% NOMES en el cas de practiques en empresa (supervisor=true) s'han de 
%% configurar les variables seguents: 

%% Supervisor del TFC/PFC 
\supervisor{Josep Jordana}

%% - Es vol incloure el logotip de l'empresa?
%%   En el cas que el TFC/PFC s'hagi fet en règim d'intercanvi amb una
%%   empresa, es pot afegir el seu logotip a la cantonada superior
%%   dreta de la portada. En aquest cas:
%%   - posar logo=true
%%   - posar el path de la imatge i l'alçada del logo a \mylogo
\setboolean{logo}{false}
\mylogo{./setup/EETAC-positiu-negre}{1.5cm}
