\section{Application Analysis}\label{S:Res:Code}
\subsection{Finite State Machine}\label{S:Res:Code:FSM}
Finite State Machine (FSM\nomenclature{FSM}{Finite State Machine}) is a computational model, usually used by hardware or some types of software, where the aplication, program or machine can have one or more different states and it will be always in one of them. This state can be changed depending on some events which are represented as an entrance on this state. Those events may, or not, generate an output and, accordingly, a new state transition.
\medskip

Figure \ref{fig:FSM} shows an easy example to represent the FSM concept. The application represented in Figure \ref{fig:FSM} can remain in three possible states while it is being executed: $S1$, $S2$ and $S3$. If application stays on $S1$, the event $E_{2}$ would imply a transition to state $S2$ but event $E_{1}$ would change the application state to $S3$. In that way, Figure \ref{fig:FSM} also shows what transitions are available and how to perform them.

\begin{figure}[h]
\centering
	\begin{subfigure}[b]{0.45\textwidth}
	\includegraphics[width=\textwidth]{FSM.eps}
	\caption{Finite State Machine}
	\label{fig:FSM}
	\end{subfigure}
	~
	\begin{subfigure}[b]{0.25\textwidth}
	\includegraphics[width=\textwidth]{Sequential.eps}
	\caption{Sequential}
	\label{fig:Sequential}
	\end{subfigure}
\caption{Programming diagrams}
\end{figure}

This concept it is easily applicable to the PIC software programmed for this report. Those applications has been based on previous programs based on sequential programming, where the application has no possibilities to move to several states depending on different events, in that case, only one transition is available as shown in Figure \ref{fig:Sequential}.
\subsection{Timers and Capture Mode}\label{S:Res:Code:Timers}
For this project, a Microchip PIC18F4520 has been used. It provides 4 Timers and a CCP\nomenclature{CCP}{Capture Compare PWM}  module. In order to count the time spend to discharge a capacitor, Timers 1 and 2 have been used for this purposes. when Timer 1 is used, it is configured as 16 bits and prescale as 1:1. If Timer 2 is set, it is also configured as 16 bits, prescale as 1:1 (8 bits configuration is not possible for Timer 2) and postcale also is 1:1. Prescale and postcale 1:1 means that counter increments every clock cycle. Timer overflow interruption is not used because timers values are going to be read once, INT0 or CCP1 interruption is thrown.

In order to detect capacitor discharge, two methods have been used during this project. First one is interruption by flag detection in port defined as INT0 and second one is the Capture Mode provided by this PIC. Capture mode introduces more accuracy during time counting because, when INT0 is used, interruption is thrown and Timer value should be requested by software, but Capture Mode stores directly the value of timer when interruption occurs. It allows to read Timer value avoiding possible undesired counts during timer reading processing.
\subsection{Code Analysis}\label{SS:Res:Code:Analysis}
As explained during introduction of this report, PIC18F4520 microcontroller has been used to analyze circuits from Figures \ref{fig:MicrocontrollerBasedInterfaceCircuitResistor2Points} and \ref{fig:MicrocontrollerBasedInterfaceCircuitResistor3Points}. As explained during Section \ref{S:Res:Analysis}, they are based on charge and discharge of RC circuit.
\medskip

Taken into account a RC circuit connected to a microcontroller pin, following states can be configured on this pin in order to either charge, discharge the capacitor $C$ or set as high-impedance input (HZ\nomenclature{HZ}{High-Impedance}). Those states can be set up modifying TRIS and PORT registers of corresponding pin port bits as defined in Table \ref{tab:PinConfigurations}.
\medskip

\begin{table}[h]
\centering
\begin{tabular}{|c|c|c|}
\hline
Pin Configuration & TRIS bit register value & PORT bit register value \\
\hline
Charge & 0 & 1 \\
\hline
Discharge & 0 & 0 \\
\hline
High-Impedance & 1 & - \\
\hline
\end{tabular}
\caption{Microcontroller pin configuration}
\label{tab:PinConfigurations}
\end{table}
\medskip

In Figures \ref{fig:MicrocontrollerBasedInterfaceCircuitResistor2Points} and \ref{fig:MicrocontrollerBasedInterfaceCircuitResistor3Points}, microcontroller pins connected to $R_{x}$, $R_{c1}$ and $R_{c2}$ should be configured as defined in Table \ref{tab:DischargePinConfigurationsRes} during capacitor discharge. Capacitor charge is performed configuring Pin 4 as Charge and Pin 1, Pin 2 and Pin 3 as High-Impedance.
\medskip

\begin{table}[h]
\centering
\begin{tabular}{|c|c|c|c|}
\hline
$N$ result & Pin 1 & Pin 2 & Pin 3 \\
\hline
$N_{x}$ & HZ & HZ & Discharge \\
\hline
$N_{c1}$ & HZ & Discharge & HZ \\
\hline
$N_{c2}$ & Discharge & HZ & HZ \\
\hline
\end{tabular}
\caption{Pin configurations during discharging time on resistive sensor system}
\label{tab:DischargePinConfigurationsRes}
\end{table}
\medskip

Pin 4 is also set to HZ during discharging process and it is configured as interruption when rising edge is detected on its port pin. It is configured on rising edge instead of falling edge (capacitor discharge) because an external Schmitt trigger is connected to Pin 4, then signal is inverted.
\medskip

Figure \ref{fig:ResistorsCodeDiagram} shows a diagram explaining the programming code of application loaded into PIC18F5420 microcontroller.
\medskip

\begin{figure}[!ht]
\centering
\includegraphics[scale=0.75]{ResistorsCodeDiagram.eps}
\caption{Block diagram describing application steps followed to measure resistor sensor}
\label{fig:ResistorsCodeDiagram}
\end{figure}

\subsection{Application using C code}\label{SS:Res:Code:Native}
Application using FSM mechanism has been used for resistor measurements. Appendix \ref{Appx:AppCodeNative} shows code written in C programming language. XC8 complier v 1.33 has been used to compile C native code and generate final *.hex file that is loaded on microcontroller.

Code allows to run application with Timer 1 or 2, depending on TIMER C macro definition value. It also allows to configure if interruption mechanism used to trigger application when capacitor is discharged by defining CCP1 C macro or not. If CCP1 is defined, CCP capture mode is used instead of INT0 interruption.

Application is waiting in a endless loop until byte with value 0x01 is received through serial port interface. As PIC18F4520 allows to configure interruption priority, Serial Port interruptions are managed as low priority interruption and INT0 or CCP1 trigger as high priority interruption. It allows that, code responsible to manage INT0 or CCP1 interruption directly disable timer currently being used, ensuring that no more timer cycles are included in timer counter. In order to avoid more instructions than required, after configure ports as required, for capacitor discharge and timer enabling, assembler (ASM\nomenclature{ASM}{Assembler}) code is used to wait until high priority interruption is processed.

Once application has collected the three required timer counters, they are sent out through serial port in order to be processed or stored for final result analysis.

\subsection{Application using Java code}\label{SS:Res:Code:Java}
As defined during this report, this project has basically three goals. First one is to show a proof of concept of how Java can work in microcontrollers with limited resources and allow to create an standard API to create interoperable Java applications. Second goal would be the analysis of direct sensor interface for resistor sensors using XC8 compiler and FSM code mechanism. In order to join these point, a third aim appear. This third point tries to compare or show the difference between create a C or Java, but application requires accuracy and maybe a specific management of some microcontroller resource. For that reason, a Java application has been created to measure resistor sensors by using the same direct interface circuit as defined in \ref{S:Res:Circuit}.

Only 3 calibration points and Timer 1 has been used to measure resistor values with Java application. The reason for this is basically that API developed for the proof of concept of this project does not allow Timer 2 configuration. In the other hand, the expected result should be worst than C native application, and 3 calibration points should return better results than 2 calibrations points, then it has been decided to just compare with best output expected results.

Java application code developed is exposed in \ref{Appx:AppCodeJava}. As defined in \ref{S:JVM:API}, Java application main code uses API defined for this project to register application to listen serial port communication. When byte is received, onEvent method is triggered and ports are configured as required in order to discharge capacitor, then application is registered to INT0 interruption. When INT0 interruption happens, onEvent method is again called, then next discharge is configured or final result is sent through serial port.

Java application cannot distinguish interruption priority, then serial port and INT0 interruption are handled inside onEvent method. Java application does not allow to implement ASM code directly, then, it is not possible to avoid more instructions than expected.