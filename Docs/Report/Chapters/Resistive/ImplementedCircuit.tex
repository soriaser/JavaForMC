\section{Implemented Circuit}\label{S:Res:Circuit}
Following concepts described in \ref{S:Res:Analysis} and \ref{S:Res:Code}, Figures \ref{fig:uR2PointsReal}, \ref{fig:uR3PointsReal}, \ref{fig:uR2PointsCCPReal} and \ref{fig:uR3PointsCCPReal} show the final circuit studied during this first part of this report.
\medskip

\begin{figure}[!ht]
\centering
	\begin{subfigure}{0.75\textwidth}
	\includegraphics[width=\textwidth]{MicrocontrollerBasedInterfaceCircuitResistor2PointsReal.eps}
	\caption{Two calibration points}
	\label{fig:uR2PointsReal}
	\end{subfigure}
	~
	\begin{subfigure}{0.85\textwidth}
	\centering
	\includegraphics[width=\textwidth]{MicrocontrollerBasedInterfaceCircuitResistor3PointsReal.eps}
	\caption{Three calibration points}
	\label{fig:uR3PointsReal}
	\end{subfigure}
\caption{Final result circuits for resistor sensors for Time capture mode}
\label{fig:FinalCircuits}
\end{figure}

\begin{figure}[!ht]
\centering
	\begin{subfigure}{0.75\textwidth}
	\centering
	\includegraphics[width=\textwidth]{MicrocontrollerBasedInterfaceCircuitResistor2PointsCCPReal.eps}
	\caption{Two calibration points}
	\label{fig:uR2PointsCCPReal}
	\end{subfigure}
	~
	\begin{subfigure}{0.85\textwidth}
	\centering
	\includegraphics[width=\textwidth]{MicrocontrollerBasedInterfaceCircuitResistor3PointsCCPReal.eps}
	\caption{Three calibration points}
	\label{fig:uR3PointsCCPReal}
	\end{subfigure}
\caption{Final result circuits for resistor sensors for CCP configuration}
\label{fig:FinalCircuitsCCP}
\end{figure}

Those circuits have been implemented using a PICDEM 2 Plus evaluation board and external breadboard. PICDEM 2 Plus has 40-pin DIP\nomenclature{DIP}{Dual In-Line Package} socket, which corresponds to PIC18F4520 (microcontroller used during this experiments). It also contains an RS-232 interface, which has been used to transmit the digital counters $N_{x}$, $N_{c1}$ and $N_{c2}$ obtained by application described in \ref{S:Res:Code}. The evaluation board has been supplied by 9V DC power supply charger because it provides a +5V regulator for direct input from 9V, 100 mA AC/DC wall adapter.
\medskip

PICDEM 2 Plus provides reset button and on-board external oscillator for 4 MHz crystal oscillator. A 4 MHz FOX F1100E oscillator has been used. As clock preescale 1:1 has been configured for the application, the counting time-base ($T_{s}$) used to get digital counts is 1 $\mu$s.
\medskip

A CD40106BE CMOS has been used as external Schmitt trigger. 4 MKT capacitors of 0.22 $\mu$F, 1 $\mu$F, 2.2 $\mu$F and 4.7 $\mu$F have been used to as $C$ in order to analyze the final result depending on capacitor value used. Following range of resistance values has been used as $R_{x}$: 825 $\Omega$, 866 $\Omega$, 909 $\Omega$, 1000 $\Omega$, 1210 $\Omega$, 1330 $\Omega$, 1400 $\Omega$, 1472 $\Omega$, 1540 $\Omega$. The real values measured for those $R_{x}$ used are: 826 $\Omega$, 863 $\Omega$, 905 $\Omega$, 1000 $\Omega$, 1208 $\Omega$, 1329 $\Omega$, 1398 $\Omega$, 1470 $\Omega$, 1539 $\Omega$. Table \ref{tab:CalibrationResUsedInR} shows values used for calibration resistors. Table \ref{tab:CalibrationResUsedInR} shows values used as calibration resistances. All of them are metallic film resistors and has been measured with ISO-TECH IDM203 bench multimeter.

\begin{savenotes}
	\begin{table}[!ht]
	\centering
		\begin{tabular}{|c|c|c|c|c|c|c|}
		\hline 
		& $R_{c1}$ ($\Omega$) & $R_{c1}$ ($\Omega$) & $R_{c2}$ ($\Omega$) & $R_{c2}$ ($\Omega$) & $R_{0}$ ($\Omega$) & $R_{0}$ ($\Omega$) \\
		\hline
		2 Points & 909 & 906 & 1330 & 1328 & 0 & 0 \\
		\hline
		3 Points & 0 & 0 & 1472 & 1470 & 340 & 339.3 \\
		\hline
		\end{tabular}
	\caption{Theoretical and Real values used for calibration resistances $R_{c1}$, $R_{c2}$ and $R_{0}$}
	\label{tab:CalibrationResUsedInR}
	\end{table}
\end{savenotes}

The values for $R_{c1}$, $R_{c2}$ and $R_{0}$ has been chosen following recommendations explained in \cite{Art:Accuracy}.