\subsection{Timers and Capture Mode}\label{S:Res:Code:Timers}
For this project, a Microchip PIC18F4520 has been used. It provides 4 Timers and a CCP\nomenclature{CCP}{Capture Compare PWM}  module. In order to count the time spend to discharge a capacitor, Timers 1 and 2 have been used for this purposes. when Timer 1 is used, it is configured as 16 bits and prescale as 1:1. If Timer 2 is set, it is also configured as 16 bits, prescale as 1:1 (8 bits configuration is not possible for Timer 2) and postcale also is 1:1. Prescale and postcale 1:1 means that counter increments every clock cycle. Timer overflow interruption is not used because timers values are going to be read once, INT0 or CCP1 interruption is thrown.

In order to detect capacitor discharge, two methods have been used during this project. First one is interruption by flag detection in port defined as INT0 and second one is the Capture Mode provided by this PIC. Capture mode introduces more accuracy during time counting because, when INT0 is used, interruption is thrown and Timer value should be requested by software, but Capture Mode stores directly the value of timer when interruption occurs. It allows to read Timer value avoiding possible undesired counts during timer reading processing.