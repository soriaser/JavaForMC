\section{Why Java for Microcontrollers?}\label{S:Introduction:Why}
Traditionally, programming languages used for microcontroller applications development have always been C or assembler. In terms of user application development, assembler required a high level of programming and memory management knowledge if a complex application is required to implement. This complexity can be decreased by using C language for these purposes, but a insufficient knowledge of it can result on bigger problems in applications developed for these kind of devices.

In the other hand, these languages do not offer the possibility to create an application interoperable in different platforms, they do not allow portability. Java is the most famous programming language that provides platform interoperability. In addition, Java is an object-oriented language with the basic features of encapsulation, polymorphism, and inheritance. Another important point is that, Java is one of programming languages used for programming and it is currently extended in universities and used by a lot of programmers and companies. Making possible to create a microcontroller applications using this programming language, it would be extensible to a higher number of people the tools and knowledge to develop any kind of project oriented to microcontroller devices. More information about programming languages popularity ca be found in \href{http://www.tiobe.com/index.php/content/paperinfo/tpci/index.html}{TIOBE website} (http://www.tiobe.com/index.php/content/paperinfo/tpci/index.html).

Already exist Java Virtual Machines developed for microcontrollers. The article \cite{Art:Darjeeling} makes a comparison between current some of them, including full standard one. All of them are focused to provide a Java standard API for specific purpose or directly implement Java Standard API as defined in \cite{Art:JVMSE7}. Taking as an example JavaCard, it is a Virtual Machine oriented to Smart Card applications development. Oracle defined a JavaCard API in order to provide methods to develop an application that is going to run on a SmartCard. These kind of applications usually use cryptography and Application Protocol Data Unit (APDU\nomenclature{APDU}{Application Protocol Data Unit}) communication to receive commands. It meas that JavaCard API provides some mechanisms to manage these kind of programming.

Using JVMs already implemented on \cite{Art:Darjeeling}, an user will be able to create an application for an specific market field, but not a generic one. For that reason, the essence of this project is born from the idea to create an standard API to expose microcontroller resources, as could be timers, ports, A/D\nomenclature{A/D}{Analogic Digital} converters, etc., as a standard Java API. Then, it facilitates to any kind of user (in this context, a programmer), to be able to create applications that are going to run over different microcontroller devices and for any purpose, regardless from API purpose.

It is well known that Java is an interpreted language, it means that performance of every instruction programmed will be affected because, it is going to be translated to an intermediate virtual machine which is the compiled one. It could introduce a delay and decrease the velocity of program execution. In microncontroller devices, not only performance could be affected, also memory and accuracy. Is that the other part of the project, create a Java application which requires high accuracy, and compare it with another one created with a compiled language like C.