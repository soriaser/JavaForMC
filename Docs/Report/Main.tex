%%%%%%%%%%%%%%%%%%%%%%%%%%%%%%%%%%%%%%%%%%%%%%%%%%%%%%%%%%%%%%%%%%%%%%%%%%%%%
%%%%%%                                                                  %%%%% 
%%%%%%          Maqueta de memòria TFC/PFC de l'EETAC                   %%%%% 
%%%%%%                                                                  %%%%% 
%%%%%%%%%%%%%%%%%%%%%%%%%%%%%%%%%%%%%%%%%%%%%%%%%%%%%%%%%%%%%%%%%%%%%%%%%%%%%
%%%%%%%%%%%%%%%%%%%%%%%%%%%%%%%%%%%%%%%%%%%%%%%%%%%%%%%%%%%%%%%%%%%%%%%%%%%%%
%%                                                                         %%
%%          Autor: Xavier Prats i Menéndez (xavier.prats@upc.edu)          %% 
%%                  Technical University of Catalonia (UPC)                %%
%%                                                                         %%
%%%%%%%%%%%%%%%%%%%%%%%%%%%%%%%%%%%%%%%%%%%%%%%%%%%%%%%%%%%%%%%%%%%%%%%%%%%%%
%%      This work is licensed under the Creative Commons  Attribution-     %%
%%   -Noncommercial-Share Alike 3.0 Spain License. To view a copy of this  %% 
%%    license, visit http://creativecommons.org/licenses/by-nc-sa/3.0/es/  %%
%%    or send a letter to Creative Commons, 171 Second Street, Suite 300,  %%
%%                  San Francisco,California, 94105, USA.                  %%
%%%%%%%%%%%%%%%%%%%%%%%%%%%%%%%%%%%%%%%%%%%%%%%%%%%%%%%%%%%%%%%%%%%%%%%%%%%%%
%% Versió 2.1 - Juliol 2012                                                %%
%%%%%%%%%%%%%%%%%%%%%%%%%%%%%%%%%%%%%%%%%%%%%%%%%%%%%%%%%%%%%%%%%%%%%%%%%%%%%

%%% NOTA: els seguents packages son necessaris per utilitzar la
%%%       plantilla seguent:
%%%       ifthen,calc,helvet,pslatex,fancyhdr,nextpage,subfigure,tocloft,graphicx,url

%%% NOTA: Es possible que algunes distribuicions Linux o Windows.
%%%       no portin aquests paquets instal·lats per defecte.
%%%       En aquest cas els haureu d'instal·lar manualment.


%%%%%%%%%%%%%%%%%%%%%%%%%%%%%%%%%%%%%%%%%%%%%%%%%%%%%%%%%%%%%%%%%%%%%%%%%%%%%
% 1- INICIALITZACIÓ
%%%%%%%%%%%%%%%%%%%%%%%%%%%%%%%%%%%%%%%%%%%%%%%%%%%%%%%%%%%%%%%%%%%%%%%%%%%%%

\documentclass[english,final]{EETAC_TFC}
%% * OPCIONS A CONFIGURAR al \documentclass
%%    - Estat del document: final o draft
%%      NOTA: Draft no inserta les figures i marca només l'espai que
%%      ocupen. També s'indica quan el text sobrepassa els marges.
%%      Draft és molt útil per compilar ràpid el document si no és important
%%      en aquell moment visualitzar les figures.
%%    - Idioma PRINCIPAL del document: catalan, spanish, english, french...

\usepackage[english,catalan]{babel}
%%  * INCLOURE TOTS ELS IDIOMES QUE S'USARAN EN EL DOCUMENT
%%    NOTA: per canviar d'idioma al mig del document usar:
%%          \selectlanguage{nom_idioma}
%%%%%%%%%%%%%%%%%%%%%%%%%%%%%%%%%%%%%%%%%%%%%%%%%%%%%%%%%%%%%%%%%%%%%%%%%%%%%

%%%%%%%%%%%%%%%%%%%%%%%%%%%%%%%%%%%%%%%%%%%%%%%%%%%%%%%%%%%%%%%%%%%%%%%%%%%%%
% 2- CÀRREGA DE PAQUETS ADICIONALS (OPCIONALS)
%%%%%%%%%%%%%%%%%%%%%%%%%%%%%%%%%%%%%%%%%%%%%%%%%%%%%%%%%%%%%%%%%%%%%%%%%%%%%

%%% NOTA: Es possible que algunes distribuicions Linux o Windows.
%%%       no portin aquests paquets instal·lats per defecte.
%%%       En aquest cas els haureu d'instal·lar manualment.

%% El paquet inputenc és extramadament útil. 
%% Permet escriure els accents directament amb l'editor de texte
%% sense haver de fer coses com per exemple: introducci\'o
%% Heu d'especificar la codificació de caracters que utilitzeu pel
%% vostre fitxer (en aquest exemple utf8)
\usepackage[utf8]{inputenc}

%% Símbols matemàtics de la American Mathematical Society
\usepackage{amssymb,amsmath,amsfonts}  

%% El paquet array proporciona eines molt útils a l'hora de fer 
%% equacions amb matrius
\usepackage{array}             

%% Paquet que permet fer taules fusionant cel·les de files consecutives
%%\usepackage{multirow}          

%% Paquet molt útil en cas de tenir taules molt llargues que 
%   ocupin vàries pàgines
%%\usepackage{longtable}          

%% Permet canviar els colors del document
%\usepackage{color,colortbl}

%% Paquet molt útil que permet activar links en el PDF final.
%% * NO OBLIDAR DE CONFIGURAR els quatre primer camps!
\usepackage[
  pdfauthor={Sergio Soria Nieto},
  pdftitle={Study and development of JVM for low and mid range microcontrollers - Sergio Soria Nieto},
  pdfsubject={Study and development of JVM for low and mid range microcontrollers},
  pdfkeywords={JVM, microcontrollers},
  pdfcreator={EETAC-UPC}, 
  pdfproducer={LaTeX, dvipdf},
  pdfdisplaydoctitle=true, plainpages=false, linktocpage=true,         
  colorlinks=true, linkcolor=blue,citecolor=blue,urlcolor=blue,
  hyperfootnotes=false, pagebackref=true, pdfpagelabels=true,
  pdfpagemode=UseOutlines,
]{hyperref} 

%% NOTA IMPORTANT!:
%% Per tal que hyperef funcioni correctament amb els capitols o seccions no
%% numerats (\chapter*{}), com per exemple introducció, conclusions i bibliografia
%% cal posar les dues comandes seguents ABANS del \chapter*{} en questió
%\cleardoublepage
%\phantomsection

%% Permet trencar links URL. 
%% Atenció! afegir aquest paquet DESPRES del hyperref!!
\usepackage{breakurl} 

%% Permet arranjar matricialment multiples figures
%% NOTA: afegir aquest paquet DESPRES del hyperref!!
%%       Si no es desitja utilitzar aquest paquet, comentar la linia seguent
%%       i anar TAMBE al fitxer de classe (eetac_tfc_pfc.cls) per substituir: 
%%       \RequirePackage[subfigure]{tocloft}  per  \RequirePackage{tocloft}
%%\usepackage{subfigmat} % I will use subcaption

%% Additional packages not included in template
\usepackage{graphicx}
\usepackage{epstopdf}
\usepackage{caption}
\usepackage{subcaption}
\usepackage{nomencl}
\usepackage{footnote}
\usepackage{listings}
\usepackage{mdframed}
\usepackage{float}

\renewcommand{\lstlistingname}{Code}
%%%%%%%%%%%%%%%%%%%%%%%%%%%%%%%%%%%%%%%%%%%%%%%%%%%%%%%%%%%%%%%%%%%%%%%%%%%%%


%%%%%%%%%%%%%%%%%%%%%%%%%%%%%%%%%%%%%%%%%%%%%%%%%%%%%%%%%%%%%%%%%%%%%%%%%%%%%
% 3- DOCUMENT
%%%%%%%%%%%%%%%%%%%%%%%%%%%%%%%%%%%%%%%%%%%%%%%%%%%%%%%%%%%%%%%%%%%%%%%%%%%%%

%%% Configuració de les dades i variables boleanes rellevants del document:
%%%%%%%%%%%%%%%%%%%%%%%%%%%%%%%%%%%%%%%%%%%%%%%%%%%%%%%%%%%%%%%%%%%%%%%%%%%%%
%%%%%%                                                                  %%%%% 
%%%%%%       Fitxer de dades per la memoria TFC/PFC de l'EETAC          %%%%% 
%%%%%%                                                                  %%%%% 
%%%%%%%%%%%%%%%%%%%%%%%%%%%%%%%%%%%%%%%%%%%%%%%%%%%%%%%%%%%%%%%%%%%%%%%%%%%%%
%%%%%%%%%%%%%%%%%%%%%%%%%%%%%%%%%%%%%%%%%%%%%%%%%%%%%%%%%%%%%%%%%%%%%%%%%%%%%
%%                                                                         %%
%%          Autor: Xavier Prats i Menendez (xavier.prats@upc.edu)          %% 
%%                  Technical University of Catalonia (UPC)                %%
%%                                                                         %%
%%%%%%%%%%%%%%%%%%%%%%%%%%%%%%%%%%%%%%%%%%%%%%%%%%%%%%%%%%%%%%%%%%%%%%%%%%%%%
%%      This work is licensed under the Creative Commons  Attribution-     %%
%%   -Noncommercial-Share Alike 3.0 Spain License. To view a copy of this  %% 
%%    license, visit http://creativecommons.org/licenses/by-nc-sa/3.0/es/  %%
%%    or send a letter to Creative Commons, 171 Second Street, Suite 300,  %%
%%                  San Francisco,California, 94105, USA.                  %%
%%%%%%%%%%%%%%%%%%%%%%%%%%%%%%%%%%%%%%%%%%%%%%%%%%%%%%%%%%%%%%%%%%%%%%%%%%%%%
%% Versio 2.1 - Juliol 2012                                                %%
%%%%%%%%%%%%%%%%%%%%%%%%%%%%%%%%%%%%%%%%%%%%%%%%%%%%%%%%%%%%%%%%%%%%%%%%%%%%%

%%%%%%%%%%%%%%%%%%%%%%%%%%%%%%%%%%%%%%%%%%%%%%%%%%%%%%%%%%%%%%%%%%%%%%%%%%%%%%%
%%  VARIABLES A CONFIGURAR                                                  %%%
%%%%%%%%%%%%%%%%%%%%%%%%%%%%%%%%%%%%%%%%%%%%%%%%%%%%%%%%%%%%%%%%%%%%%%%%%%%%%%%

%% - Projecte o Treball de Fi de Carrera?
%%      PFC = true   -> Projecte de Fi de Carrera
%%      PFC = false  -> Treball  de Fi de Carrera
\setboolean{PFC}{false}

%% - Escollir la titulació
%\titulacio{Enginyeria Tècnica Aeronàutica, especialitat Aeronavegació}
%\titulacio{Enginyeria T\`ecnica de Telecomunicaci\'o, especialitat Sistemes de Telecomunicaci\'o}
%\titulacio{Enginyeria T\`ecnica de Telecomunicaci\'o, especialitat Telem\`atica}
%\titulacio{Enginyeria de Telecomunicaci\'o (segon cicle)}
% Modificació respecte a la versió 2.1 - Iván Padilla Montero - Juliol 2014
%\titulacio{Grau en Enginyeria d'Aeronavegaci\'o}
%\titulacio{Grau en Enginyeria d'Aeroports}
%\titulacio{Grau en Enginyeria Telemàtica}
%\titulacio{Grau en Enginyeria de Sistemes de Telecomunicació}
\titulacio{Master's degree of Science in Telecommunication Engineering and Management}


%% - Configurar els idiomes del document
%% Si l'idioma PRINCIPAL del document es l'angles, posar aquesta variable a true
\setboolean{Leng}{true}

%% Escollir entre catala i castella (idioma principial, o nomes pel resum en cas que l'idioma principal sigui anglès)
%%  catala = true   -> idioma principal (o només resum) en Català
%%  catala = false  -> idioma principal (o només resum) en Castella
\setboolean{Lcat}{true}

%% Titol del document en l'idioma principal del document 
\titol{Study and development of JVM for low and mid range microcontrollers}

%% Titol del document en anglès (Per l'apartat overview)
\titolE{Study and development of JVM for low and mid range microcontrollers}

%% Titol del document en catala/castella (Per l'apartat resum)
\titolC{Study and development of JVM for low and mid range microcontrollers}


%% - Nombre d'autors del TFC/PFC?
%%      UNautor = true   Un sol autor
%%      UNautor = false  Més d'un autor
\setboolean{UNautor}{true}

%% - Nom del(s) Autor(s) del document
%% NOTA: En cas de mes d'un autor cal posar la comana \and entre els
%%        noms dels autors
\autor{Sergio Soria Nieto}

%% - Nombre de directors del TFC/PFC. Tipicament 1 o 2
%%      UNdirector = true   Un sol director
%%      UNdirector = false  Dos directors
\setboolean{UNdirector}{true}

%% - Nom del Director del TFC/PFC
\director{Josep Jordana}

%% - Nom del segon director en cas de tenir-lo:
\segonDirector{Nom2 Cognoms2}


%% - Es vol incloure una dedicatoria?
%%      dedicatoria = true   -> S'afegeix una pagina amb \textDedicatoria
%%      dedicatoria = true   -> No s'afegeix dedicatoria
%% NOTA: no confondre dedicatòria amb agraïments. Una dedicatoria sol ser
%%       un missatge curt d'una o dues frases màxim a la persona, o persones
%%       a les quals es dedica el treball. 
%%       Els agraïments poden ser extensos i l'autor pot agraïr a diverses
%%       persones coses diferents en funció de l'ajuda rebuda, per exemple. 
%%       Si es volen incloure agraïments, fer-ho al fitxer de la 
%%       memòria creant una secció nova amb  \chapter*{Agraïments}
\setboolean{dedicatoria}{true}
\textDedicatoria{TODO}

%% - Es vol incloure una pagina d'index de figures?
\setboolean{paginaLOF}{true}  % List of Figures

%% - Es vol incloure una pagina d'index de taules?
\setboolean{paginaLOT}{true}  % List of Tables 

%% - El projecte ha estat supervisat per alguna persona externa? 
%%   (NOMES en cas de practiques en empresa)
%%      supervisor = true    -> Hi ha un supervisor
%%      supervisor = false   -> No hi ha un supervisor
\setboolean{supervisor}{true}

%% NOMES en el cas de practiques en empresa (supervisor=true) s'han de 
%% configurar les variables seguents: 

%% Supervisor del TFC/PFC 
\supervisor{Josep Jordana}

%% - Es vol incloure el logotip de l'empresa?
%%   En el cas que el TFC/PFC s'hagi fet en règim d'intercanvi amb una
%%   empresa, es pot afegir el seu logotip a la cantonada superior
%%   dreta de la portada. En aquest cas:
%%   - posar logo=true
%%   - posar el path de la imatge i l'alçada del logo a \mylogo
\setboolean{logo}{false}
\mylogo{./setup/EETAC-positiu-negre}{1.5cm}


%%% Configuració de MACROS o ENTORNS (opcionals) definides per l'usuari:
%%%%%%%%%%%%%%%%%%%%%%%%%%%%%%%%%%%%%%%%%%%%%%%%%%%%%%%%%%%%%%%%%%%%%%%%%%%%%
%%%%%%                                                                  %%%%% 
%%%%%%    Fitxer de macros d'usuari per la memoria TFC/PFC de l'EETAC   %%%%% 
%%%%%%                                                                  %%%%% 
%%%%%%%%%%%%%%%%%%%%%%%%%%%%%%%%%%%%%%%%%%%%%%%%%%%%%%%%%%%%%%%%%%%%%%%%%%%%%
%%%%%%%%%%%%%%%%%%%%%%%%%%%%%%%%%%%%%%%%%%%%%%%%%%%%%%%%%%%%%%%%%%%%%%%%%%%%%
%%                                                                         %%
%%         Author: Xavier Prats i Menendez (xavier.prats@upc.edu)          %% 
%%                  Technical University of Catalonia (UPC)                %%
%%                                                                         %%
%%%%%%%%%%%%%%%%%%%%%%%%%%%%%%%%%%%%%%%%%%%%%%%%%%%%%%%%%%%%%%%%%%%%%%%%%%%%%
%%      This work is licensed under the Creative Commons  Attribution-     %%
%%   -Noncommercial-Share Alike 3.0 Spain License. To view a copy of this  %% 
%%    license, visit http://creativecommons.org/licenses/by-nc-sa/3.0/es/  %%
%%    or send a letter to Creative Commons, 171 Second Street, Suite 300,  %%
%%                  San Francisco,California, 94105, USA.                  %%
%%%%%%%%%%%%%%%%%%%%%%%%%%%%%%%%%%%%%%%%%%%%%%%%%%%%%%%%%%%%%%%%%%%%%%%%%%%%%
%% Versio 1.5 - Juliol 2010                                                %%
%%%%%%%%%%%%%%%%%%%%%%%%%%%%%%%%%%%%%%%%%%%%%%%%%%%%%%%%%%%%%%%%%%%%%%%%%%%%%


%%% Xevi's macros for vectors and matrices:

%\newcommand{\ve}[1]{\mbox{\boldmath$#1$}}          
\newcommand{\ve}[1]{\vec{#1}}  
\newcommand{\ma}[1]{\mbox{\boldmath$\mathcal{#1}$}}

%%% Xevi's macros for brackets:
\newcommand{\lp}{\left(}
\newcommand{\lc}{\left[}
\newcommand{\lcl}{\left\{}
\newcommand{\rp}{\right)}
\newcommand{\rc}{\right]}
\newcommand{\rcl}{\right\}}

%%% Xevi's new environment for HIPOTESIS
\newcounter{num_hyp}
\newenvironment{hyp}[2]{
        \refstepcounter{num_hyp}
        \vspace*{2.5ex}
        {\noindent \bf\sffamily HYPOTHESIS #1 : #2} \\
        \sl
}
        {\vspace{1ex}
}

\newcommand{\SUMhyp}[2]{
 {\sffamily HYPOTHESIS #1 : #2} 
}


%%% Configuració manual de les regles d'hyphenation:
%%%%%%%%%%%%%%%%%%%%%%%%%%%%%%%%%%%%%%%%%%%%%%%%%%%%%%%%%%%%%%%%%%%%%%%%%%%%%
%%%%%%                                                                  %%%%% 
%%%%%%    Fitxer de hyphenation per la memoria TFC/PFC de l'EETAC       %%%%% 
%%%%%%                                                                  %%%%% 
%%%%%%%%%%%%%%%%%%%%%%%%%%%%%%%%%%%%%%%%%%%%%%%%%%%%%%%%%%%%%%%%%%%%%%%%%%%%%
%%%%%%%%%%%%%%%%%%%%%%%%%%%%%%%%%%%%%%%%%%%%%%%%%%%%%%%%%%%%%%%%%%%%%%%%%%%%%
%%                                                                         %%
%%         Author: Xavier Prats i Menendez (xavier.prats@upc.edu)          %% 
%%                  Technical University of Catalonia (UPC)                %%
%%                                                                         %%
%%%%%%%%%%%%%%%%%%%%%%%%%%%%%%%%%%%%%%%%%%%%%%%%%%%%%%%%%%%%%%%%%%%%%%%%%%%%%
%%      This work is licensed under the Creative Commons  Attribution-     %%
%%   -Noncommercial-Share Alike 3.0 Spain License. To view a copy of this  %% 
%%    license, visit http://creativecommons.org/licenses/by-nc-sa/3.0/es/  %%
%%    or send a letter to Creative Commons, 171 Second Street, Suite 300,  %%
%%                  San Francisco,California, 94105, USA.                  %%
%%%%%%%%%%%%%%%%%%%%%%%%%%%%%%%%%%%%%%%%%%%%%%%%%%%%%%%%%%%%%%%%%%%%%%%%%%%%%
%% Versio 1.5 - Juliol 2010                                                %%
%%%%%%%%%%%%%%%%%%%%%%%%%%%%%%%%%%%%%%%%%%%%%%%%%%%%%%%%%%%%%%%%%%%%%%%%%%%%%

\hyphenation{Cas-tell-de-fels}
\hyphenation{EETAC}



\makenomenclature

\graphicspath{{Images/}}
\DeclareGraphicsExtensions{.eps}

\begin{document}

%% Seleccionar l'idioma principal del document:
\selectlanguage{english}

\beforepreface  

%% RESUM i OVERVIEW
%%%%%%%%%%%%%%%%%%%%%%%%%%%%%%%%%%%%%%%%%%%%%%%%%%%%%%%%%%%%%%%%%%%%%%%%%%%%%
% NOTA: les longituds passades com a parametres d'entrada  s'han
%        d'ajustar manualment fins que el requadre del resum/overview
%        ocupi tota la pàgina. 

%%% Resum en anglès
\selectlanguage{english}   
\begin{overview}{11cm}
The aim of this project is to develop a Java Virtual Machine and design Java specific API to provide a easy mechanism to use microcontroller resources using similar concept used in JavaCard for SmartCards.

The main idea is to implement a JVM which could be executed in microcontrollers based on 1 kB RAM or more. This JVM is developed in C and provides a Java API to get access to microcontroller resources (ports, timers, serial port, etc.). It allows to implement microcontrollers Java applications and that means support to object oriented programming and compatible with any microcontroller running this JVM on it, i.e. one Java application that will work in all microcontrollers.

This JVM should be a reduced version of standard JVM in order to successfully execute it on low and mid range microncontroller with limited resources. This project is based on JavaCard for SmartCards, for that reason, an introduction and comparison with this technology and how it is currently used on real life using high range microcontrollers will be analyzed. JVM will be initially developed to PIC18F4520 and PIC16F87X and adaptation to other devices is out of the scope an open for future works.

Then, final goal is to make a comparison, in terms of performance and accuracy, between applications based on this JVM and native application. Regarding accuracy, resistive capture method by microcontroller PIC18F4520 will be analized. Regarding performance has to be defined.
\end{overview}

% Tornar a l'idioma principal del document
\selectlanguage{english}  

%NOTA: En cas d'utilitzar l'espanyol com a idioma principal del document, el
%      latex anomena les taules com a 'Cuadros'. Si es desitja canviar aquesta
%      nomenclatura i utilitzar la paraula 'Tabla' descomentar les línies següents:
%\def\listtablename{Índice de tablas}
%\def\tablename{Tabla}%

\printnomenclature

% Amb aqueta comanda indiquem que ja s'han inclòs tots els apartats del prefaci del 
% projecte o podem començar a incloure els capitols de la memòria
\afterpreface


%%%%%%%%%%%%%%%%%%%%%%%%%%%%%%%%%%%%%%%%%%%%%%%%%%%%%%%%%%%%%%%%%%%%%%%%%%
%%%%%% INCLOURE A PARTIR D'AQUÍ TOTS ELS CAPÍTOLS DE LA MEMORIA   %%%%%%%%
%%%%%%%%%%%%%%%%%%%%%%%%%%%%%%%%%%%%%%%%%%%%%%%%%%%%%%%%%%%%%%%%%%%%%%%%%%

% NOTA: recordar que la introducció i les conclusions són capítols NO
%       enumerats, per tant s'ha d'usar \chapter*

% NOTA: és aconsellable incloure els capítols de la memòria en fitxers 
%       separats utlitzant la comanda \input  Per exemple:
%       \input{capitol1}  
%       que farà que s'inclogui el fitxer capitol1.tex

% NOTA: Si es vol incloure agraïments i/o glosari, fer-ho utilitzant 
% \chapter*{} i incloure'ls abans la introducció

\chapter{Introduction}\label{C:Introduction}
The aim of this project is to develop a Java Virtual Machine and design Java specific API\nomenclature{API}{Application Programming Interface} to provide a easy mechanism to use microcontroller resources using similar concept used in JavaCard for SmartCards.

The main idea is to implement a JVM\nomenclature{JVM}{Java Virtual Machine} which could be executed in microcontrollers based on 1 kB RAM\nomenclature{RAM}{Random Access Memory} or more. This JVM is developed in C and provides a Java API to get access to microcontroller resources (ports, timers, serial port, etc.). It allows to implement microcontrollers Java applications and that means support to object oriented programming and compatible with any microcontroller running this JVM on it, i.e. one Java application that will work in all microcontrollers.

This JVM should be a reduced version of standard JVM in order to successfully execute it on low and mid range microncontroller with limited resources. This project is based on JavaCard for SmartCards, for that reason, an introduction and comparison with this technology and how it is currently used on real life using high range microcontrollers will be analyzed. JVM will be initially developed to PIC18F4520 and PIC16F87X and adaptation to other devices is out of the scope an open for future works.

In order to check the behavior of Java for microcontrollers designed for this project in accuracy environments, during second part of this report is going to be analyzed a direct interface circuit for resistive sensors by developing a native application in C based on \cite{Art:Accuracy} methods. The same application is developed in Java by using API and JVM described and designed in this report. Then, final goal is to make a comparison, in terms accuracy, between applications based on this Java and native application. Regarding accuracy, resistive capture method by microcontroller PIC18F4520 \nomenclature{PIC}{Programmable Interrupt Controller} will be analized.

First chapter exposes the motivation of this project and why it is interesting to introduce a programming language like Java in a microcnotrollers world. Existing preceding implementations of VM\nomenclature{VM}{Virtual Machine} are compared and why it is required a new one. Second chapter explains the complete design of JVM developed and how it works. During this chapter, the process since a Java application is created until it starts to run is explained in details. Then, third chapter introduces the theory related to direct interface circuits for resistive sensors and different implementations of an application developed for this, one in C and other one in Java, are compared. Finally, report concludes with describing conclusions extracted from results obtained.

\chapter{Why Java for Microcontrollers?}\label{C:Why}
Traditionally, programming languages used for microcontroller applications development have always been C or assembler. In terms of user application development, assembler required a high level of programming and memory management knowledge if a complex application is required to implement. This complexity can be decreased by using C language for these purposes, but a insufficient knowledge of it can result on bigger problems in applications developed for these kind of devices.

In the other hand, these languages do not offer the possibility to create an application interoperable in different platforms, they do not allow portability. Java is the most famous programming language that provides platform interoperability. In addition, Java is an object-oriented language with the basic features of encapsulation, polymorphism, and inheritance. Another important point is that, Java is one of programming languages used for programming and it is currently extended in universities and used by a lot of programmers and companies. Making possible to create a microcontroller applications using this programming language, it would be extensible to a higher number of people the tools and knowledge to develop any kind of project oriented to microcontroller devices. More information about programming languages popularity ca be found in \href{http://www.tiobe.com/index.php/content/paperinfo/tpci/index.html}{TIOBE website}.

Already exist Java Virtual Machines developed for microcontrollers. The article \cite{Art:Darjeeling} makes a comparison between current some of them, including full standard one. All of them are focused to provide a Java standard API for specific purpose or directly implement Java Standard API as defined in \cite{Art:JVMSE7]}. Taking as an example JavaCard, it is a Virtual Machine oriented to Smart Card applications development. Oracle defined a JavaCard API in order to provide methods to develop an application that is going to run on a SmartCard. These kind of applications usually use cryptography and Application Protocol Data Unit (APDU\nomenclature{APDU}{Application Protocol Data Unit}) communication to receive commands. It meas that JavaCard API provides some mechanisms to manage these kind of programming.

Using JVMs already implemented on \cite{Art:Darjeeling}, an user will be able to create an application for an specific market field, but not a generic one. For that reason, the essence of this project is born from the idea to create an standard API to expose microcontroller resources, as could be timers, ports, A/D converters, etc., as a standard Java API. Then, it facilitates to any king of user (in this context, a programmer), to be able to create applications that are going to run over different microcontroller devices and for any purpose, regardless from API purpose.

It is well known that Java is an interpreted language, it means that performance of every instruction programmed will be affected because, it is going to be translated to an intermediate virtual machine which is the compiled one. It could introduce a delay and decrease the velocity of program execution. In microncontroller devices, not only performance could be affected, also memory and accuracy. Is that the other part of the project, create a Java application which requires high accuracy, and compare it with another one created with a compiled language like C.

\chapter{Why Java for Microcontrollers?}\label{C:Why}
Traditionally, programming languages used for microcontroller applications development have always been C or assembler. In terms of user application development, assembler required a high level of programming and memory management knowledge if a complex application is required to implement. This complexity can be decreased by using C language for these purposes, but a insufficient knowledge of it can result on bigger problems in applications developed for these kind of devices.

In the other hand, these languages do not offer the possibility to create an application interoperable in different platforms, they do not allow portability. Java is the most famous programming language that provides platform interoperability. In addition, Java is an object-oriented language with the basic features of encapsulation, polymorphism, and inheritance. Another important point is that, Java is one of programming languages used for programming and it is currently extended in universities and used by a lot of programmers and companies. Making possible to create a microcontroller applications using this programming language, it would be extensible to a higher number of people the tools and knowledge to develop any kind of project oriented to microcontroller devices. More information about programming languages popularity ca be found in \href{http://www.tiobe.com/index.php/content/paperinfo/tpci/index.html}{TIOBE website}.

Already exist Java Virtual Machines developed for microcontrollers. The article \cite{Art:Darjeeling} makes a comparison between current some of them, including full standard one. All of them are focused to provide a Java standard API for specific purpose or directly implement Java Standard API as defined in \cite{Art:JVMSE7]}. Taking as an example JavaCard, it is a Virtual Machine oriented to Smart Card applications development. Oracle defined a JavaCard API in order to provide methods to develop an application that is going to run on a SmartCard. These kind of applications usually use cryptography and Application Protocol Data Unit (APDU\nomenclature{APDU}{Application Protocol Data Unit}) communication to receive commands. It meas that JavaCard API provides some mechanisms to manage these kind of programming.

Using JVMs already implemented on \cite{Art:Darjeeling}, an user will be able to create an application for an specific market field, but not a generic one. For that reason, the essence of this project is born from the idea to create an standard API to expose microcontroller resources, as could be timers, ports, A/D converters, etc., as a standard Java API. Then, it facilitates to any king of user (in this context, a programmer), to be able to create applications that are going to run over different microcontroller devices and for any purpose, regardless from API purpose.

It is well known that Java is an interpreted language, it means that performance of every instruction programmed will be affected because, it is going to be translated to an intermediate virtual machine which is the compiled one. It could introduce a delay and decrease the velocity of program execution. In microncontroller devices, not only performance could be affected, also memory and accuracy. Is that the other part of the project, create a Java application which requires high accuracy, and compare it with another one created with a compiled language like C.
\chapter{Java Virtual Machine}\label{C:JVM}
When a java application is compiled and generates an output file for every java object. This file is named class file. The class file contains different required information to be loaded on a JVM\nomenclature{JVM}{Java Virtual Machine}, as defined in \cite{Art:JVMSE7}, and execute the java code. This format was designed for large RAM memory devices and it has been redesign for this project. This new design is based on NanoVM \cite{Art:NanoVM}.

\section{Design}\label{S:JVM:Design}
The design for a new JVM adapted for $\mu$C\nomenclature{$\mu$C}{Microcontroller} implies several entities external to $\mu$C final code running on it. It requires a full system since java application is created, compiled and loaded in JVM running in $\mu$C. It also requires a design of Java API\nomenclature{API}{Application Programming Interface} responsible to provide access to $\mu$C resources, which are only available if application is programmed in C or assembler.

Diagram of Figure \ref{fig:C:JVM:OverviewDiagram} shows all parts that take part of full process. The main idea is go through those parts explaining in detail every process reflected in this diagram. Dark boxes of this figure indicates the proprietary parts of this project (i.e. fully designed and implemented for this document purpose), while clear boxes are standard or already existing parts that have been used in this project.

\begin{figure}[H]
\centering
\includegraphics[scale=0.5]{DiagramOverviewDesign.eps}
\caption{Overview of full design implemented on this project in order to provide and environment to execute java applications in $\mu$Cs}
\label{fig:C:JVM:OverviewDiagram}
\end{figure}

Looking at Figure \ref{fig:C:JVM:OverviewDiagram} it is possible to distinguish two paths finishing on $\mu$C. Right path describes the part responsible to create and load the main system that is going to run into $\mu$C. This system or OS\nomenclature{OS}{Operating System} is going to be loaded just one time during life cycle of the device. It is able to load, delete and interpret following Java applications that will be loaded and deleted several times. Then, left path indicates all steps required to load a final Java application that is going be executed on $\mu$C (i.e. interpreted by JVM loaded on the device). It means that, the first and mandatory process is download the OS. After that, it is possible to load or delete the Java applications several times.

\section{API}\label{S:JVM:API}
One of the objectives of this project is to provide an easy way to create microcontroller applications for those users without high experience in embedded programming or memory management. Java makes it easier, but it is required to provides a mechanism to access, control and use the several resources like Timers, Pins, etc. When a user programs one application in C or assembler, it is easy to modify microcontroller registers for application purposes, then, it is required a Java API\nomenclature{API}{Application Programming Interface} responsible to provide a maximum range of methods to provide, as much as possible, the same possibilities as user programs in C or assembler.

Following API has been design for this project has been divided in following Java packages:

\begin{itemize}
\item java.mc. Main package for $\mu$C application.
\item java.mc.ports. Package containing methods to use $\mu$C pin ports.
\item java.mc.serialport. Package containing methods to use $\mu$C serial port.
\item java.mc.timers. Package containing methods to use $\mu$C timers.
\end{itemize}

\subsection{java.mc}\label{SS:JVM:API:Main}
All standard Java objects inherit the also standard main root class \textit{java.lang.Object}. It means that, when you create an standard Java application, although it is not explicitly written as example \ref{cod:SS:JVM:API:Main:StdJavaMainHeader}, the class is based on or extends from \textit{java.lang.Object}. The application starting point is the main method as shown in example \ref{cod:SS:JVM:API:Main:StdJavaMain}.

\medskip
\begin{lstlisting}[
caption={Standard main Java class real header},
captionpos=b,
label={cod:SS:JVM:API:Main:StdJavaMainHeader},
language=Java]
public class Main extends java.lang.Object
\end{lstlisting}
\medskip

\medskip
\begin{lstlisting}[
caption={Standard main Java class},
captionpos=b,
label={cod:SS:JVM:API:Main:StdJavaMain},
language=Java]
package my.java.app;

public class Main {

    public static void main(String[] args) {
        // code...
    }

}
\end{lstlisting}
\medskip

The package java.mc includes two java classes: \textit{Microapplication} and \textit{MicroapplicationListener}. A Java application developed to run in the system described in this paper should extend \textit{Microapplication} class instead of \textit{java.lang.Object}. It means that it inherits all methods from this class as described in \ref{cod:SS:JVM:API:Main:ClassMicroApp}.

\medskip
\begin{lstlisting}[
caption={Microapplication Class},
captionpos=b,
label={cod:SS:JVM:API:Main:ClassMicroApp},
language=Java]
package java.mc;

public abstract class MicroApplication {

    /**
     * Constructor.
     */
    protected MicroApplication() {}

    /**
     * Starting point of microcontroller application. After
     * device power up or reset, JVM integrated in device
     * starts running this method.
     */
    public abstract void main();

    /**
     * Some interruptions can happen during life cycle of the
     * device, and it is possible that the application
     * requires to make some process when it this interruption
     * happens. This interruptions are called events in this
     * case.
     * 
     * SetEvent provides a mechanism to register the
     * application to one specific event, while ClearEvent
     * allows to deregister the events. 
     * 
     * @param event Event to register. Current possible
     * options are:
     * PortConstants.EVENT_INTERRUPT_0,
     * PortConstants.EVENT_INTERRUPT_1 or
     * SerialPortConstants.EVENT_RECEIVED_BYTE
     */
    public static native void ClearEvent(byte event);

    /**
     * Some interruptions can happen during life cycle of the
     * device, and it is possible that the application
     * requires to make some process when it this interruption
     * happens. This interruptions are called events in this
     * case.
     * 
     * This method provides a mechanism to register the
     * application to one specific event. If application
     * requires to be registered to several events, it has to
     * be called several times indicating different events.
     * 
     * If application is registered to some event, it has to
     * implement MicroApplicationListener interface because,
     * if the event happens, onEvent method is called. 
     * 
     * @param event Event to register. Current possible
     * options are:
     * PortConstants.EVENT_INTERRUPT_0,
     * PortConstants.EVENT_INTERRUPT_1 or
     * SerialPortConstants.EVENT_RECEIVED_BYTE
     */
    public static native void SetEvent(byte event);

    /**
     * Sets device in sleep mode during some milliseconds.
     * During this time, the microcontroller is not executing
     * any code.
     *
     * @param milliseconds Number of millisecons to wait.
     */
    public static native void Sleep(short milliseconds);

}
\end{lstlisting}
\medskip

\textit{MicroapplicationListener} is a Java interface that includes method \textit{onEvent}. It means that any Java class that implements this interface has to implement this method. \textit{onEvent} method is called every time an event registered by \textit{ClearEvent} is triggered in the device. These events are basically interruption on pins, timers or similar. Code example \ref{cod:SS:JVM:API:Main:ClassReceive} shows an application registered to an event. In this case the event is receive one byte by Serial Port pins.

\medskip
\begin{lstlisting}[
caption={Example on microapplication Java applicaiton registered to device event},
captionpos=b,
label={cod:SS:JVM:API:Main:ClassReceive},
language=Java]
public class Receive extends MicroApplication implements
MicroApplicationListener {

    public void main() {
        // Application initially is registered to
        // EVENT_RECEIVED_BYTE event in order to trigger
        // onEvent method every time a byte is received
        // through Serial Port.
        MicroApplication.SetEvent(
            SerialPortConstants.EVENT_RECEIVED_BYTE);
    }

    public void onEvent(byte event) {
    	if (SerialPortConstants.EVENT_RECEIVED_BYTE
    	    == event) {
    		// Code...
    	}
    }

}
\end{lstlisting}
\medskip

\subsection{java.mc.ports}\label{SS:JVM:API:Ports}
One of the main resources of a $\mu$C are the ports, where it is possible to control one pin of this port as an input or output, and set its output state to '0' or '1'. For that reason, \textit{Port} Java class has been designed to control those states. In addition, sometimes, $\mu$C provides several integrated ports with their respective pins, then a method to request which port is going to be used has been also added to this Port Java Class as defined in \ref{cod:SS:JVM:API:Ports:Class}. \textit{PortConstants} is an interface to provide several constants available.

\medskip
\begin{lstlisting}[
caption={Port Class},
captionpos=b,
label={cod:SS:JVM:API:Ports:Class},
language=Java]
package java.mc.ports;

public final class Port {

    /**
     * Some microcontrollers integrates several ports with
     * their pins to manipulate. This method returns a Port
     * class representing this device resource but allows to
     * specify which port is going to be requested.
     *
     * @param port Device port to be requested. Options can
     * be:
     * PortConstants.PORTA, PortConstants.PORTB or
     * PortConstants.PORTC.
     *
     * @return Port class representing microcontroller
     * resource.
     */
    public native static Port getPort(byte port);

    /**
     * Sets specific pin of port to logical '0'.
     *
     * @param pin Pin to set.
     */
    public native void setPinToZero(byte pin);

    /**
     * Sets specific pin of port to logical '1'.
     *
     * @param pin Pin to set.
     */
    public native void setPinToOne(byte pin);

    /**
     * Bit mask indicating pins to set to '1' or '0'.
     * Bits sets to '1' in bit mask will set real output pin
     * to '1'.
     * Bits sets to '0' in bit mask will set real output pin
     * to '0'.
     * 
     * @param pin Byte representing output pins state.
     * bit X -> pin X.
     */
    public native void setPins(byte pins);

    /**
     * Sets specific pin of port as input.
     *
     * @param pin Pin to set.
     */
    public native void setInputPin(byte pin);

    /**
     * Sets specific pin of port as output.
     *
     * @param pin Pin to set.
     */
    public native void setOutputPin(byte pin);

    /**
     * Bit mask indicating pins directions.
     * Bits sets to '1' in bit mask will set pin as input.
     * Bits sets to '0' in bit mask will set pin as output.
     * 
     * @param pin Byte representing pins directions.
     * bit X -> pin X.
     */
    public native void setIO(byte directions);

}
\end{lstlisting}
\medskip

\subsection{java.mc.serialport}\label{SS:JVM:API:SerialPort}
Some microcontrollers allow configuration of some pins in order to receive data by Serial Port connected to some specific pins. This Java package contains a functionality to facilitate how to receive and send data by Serial Port. It has been   pulled out from ports packages \ref{SS:JVM:API:Ports} because it has been considered an specific configuration an functionality of the device even if it is related to ports. Usually, enable and set parameters related to this kind of configuration does not only involve modification of registers related to pins.

\medskip
\begin{lstlisting}[
caption={SerialPort Class},
captionpos=b,
label={cod:SS:JVM:API:SerialPort:Class},
language=Java]
package java.mc.serialport;

public final class SerialPort {

    /**
     * If byte has been received through Serial Port, it
     * returns the value of it. If byte has not been received,
     * it returns current value in receive register of
     * microcontroller.
     *
     * @return Received byte value.
     */
    public native static byte Receive();

    /**
     * Sends some data through SerialPort pins if they are
     * connected.
     *
     * @param data Bytes to send.
     * @param offset Offset in byte array where data to
     * send starts.
     * @param length Number of bytes to send.
     */
    public native static void Send(byte[] data, short offset,
      short length);

}
\end{lstlisting}
\medskip

\textit{SerialPortConstants} is an interface to provide several constants related to Serial Port. Currently, only constant EVENT\_ RECEIVED\_ BYTE exists and it is an event id to use by \textit{SetEvent()} in order to through \textit{onEvent()} method if byte is received through Serial Port. Method \textit{send()} does not require any registration.

\subsection{java.mc.timers}\label{SS:JVM:API:Timers}
One of the common resources of a $\mu$C are the timers. They can be used for several purposes. Microcontrollers usually can provide several Timers, every one with their possible configurations an characteristics. This Java package tries to provide an API to control them. Currently, configuration is very limited because it has been only defined an developed for this project purposes, but it is enough to show that it is possible to create a Java class that represents every resource available in $\mu$Cs and manage them.

\medskip
\begin{lstlisting}[
caption={Timer Class},
captionpos=b,
label={cod:SS:JVM:API:Timers:Class},
language=Java]
package java.mc.timers;

public final class Timer {

    /**
     * Some microcontrollers integrates several timers to configurate.
     * This method returns a Timer class representing this device resource
     * but allows to specify which timer is going to be requested.
     *
     * @param timer Device timer to be requested. Options can
     * be:
     * TimerConstants.TIMER0 and TimerConstants.TIMER1.
     *
     * @return Timer class representing microcontroller
     * resource.
     */
    public native static Timer getTimer(byte timer);

    /**
     * Clears current value of timer counter register.
     */
    public native void clear();

    /**
     * Disables timer.
     */
    public native void disable();

    /**
     * Enables timer.
     */
    public native void enable();

    /**
     * Reads current value of timer register.
     */
    public native short read();

}
\end{lstlisting}
\medskip

\textit{TimertConstants} is an interface to provide several constants related to Tiemr. Currently, only constants related to interruptions are not available but it can be considered for future API extensions.

\section{JCR}\label{S:JVM:JCR}
Any standard Java application, once it is compiled, a new file \textit{*.class} is created. The structure of this file is explained in detail in chapter 4 of \cite{Art:JVMSE7}. The standard format is designed for high RAM capacity devices, while $\mu$Cs are very limited in that sense. Then, JCR\nomenclature{JCR}{Java Class Reduced} format has been designed for this project and for Java microcontroller purposes. This new structure is based on NanoVM \cite{Art:NanoVM} defined structure.

This chapter defines its own set of data types representing JCR file data. types \textit{uint8\_ t} and \textit{uint16\_ t} represent an unsigned eight- or sixteen-bits, respectively. Symbol * represents a pointer of this type. JCR file consists of following structure:

\medskip
\begin{lstlisting}[
caption={JCR file structure},
captionpos=b,
label={cod:S:JVM:JCR:JCRStruct},
language=C]
typedef struct jcr {
  jcr_header header;
  jcr_class* classes;
  uint8_t* constants;
  jcr_method_header* methods;
  uint8_t* bytecodes;
} jcr_t;
\end{lstlisting}
\medskip

\medskip
\begin{description}
\item[jcr\_ header] \hfill \\
Required information as defined in \ref{cod:S:JVM:JCR:JCRHeaderStruct}.
\item[javaclass\_ class] \hfill \\
Array of Java Classes loaded in this JCR application.
\item[constants] \hfill \\
Identifiers of global constants that are going to be stored in stack.
\item[javaclass\_ method\_ header] \hfill \\
Array of methods loaded as a sum of all methods used in every Java Class.
\item[bytecodes] \hfill \\
Translated bytecodes containing all methods instructions to execute.
\end{description}
\medskip

\medskip
\begin{lstlisting}[
caption={JCR Header structure},
captionpos=b,
label={cod:S:JVM:JCR:JCRHeaderStruct},
language=C]
typedef struct jcr_header {
  uint8_t  version;
  uint8_t  methods;
  uint16_t constants;
  uint8_t  classes;
  uint8_t  fields;
  uint8_t  main;
  uint16_t offsetClasses;
  uint16_t offsetConstants;
  uint16_t offsetMethods;
} jcr_header_t;
\end{lstlisting}
\medskip

\begin{description}
\item[version] \hfill \\
JCR version. Currently, 0x01 is used.
\item[methods] \hfill \\
Number of methods by taking into account all Java Classes.
\item[constants] \hfill \\
Number of constants by taking into account all Java Classes.
\item[classes] \hfill \\
Number of Java Classes.
\item[fields] \hfill \\
Number of fields by taking into account all Java Classes.
\item[main] \hfill \\
Offset within JCR file to the main() Java method as defined in \ref{SS:JVM:Micros:API}.
\item[offsetClasses] \hfill \\
Offset within JCR file to loaded Java Classes.
\item[offsetConstants] \hfill \\
Offset within JCR file to loaded constants.
\item[offsetMethods] \hfill \\
Offset within JCR file to loaded methods.
\end{description}

\begin{lstlisting}[
caption={JCR Class structure},
captionpos=b,
label={cod:S:JVM:JCR:JCRClassStruct},
language=C]
typedef struct jcr_class {
  uint8_t super;
  uint8_t fields;
} jcr_class_t;
\end{lstlisting}

\begin{description}
\item[super] \hfill \\
Indicates the class identifier of its super class.
\item[fields] \hfill \\
Number of fields belonging to this class.
\end{description}

\begin{lstlisting}[
caption={JCR Method structure},
captionpos=b,
label={cod:S:JVM:JCR:JCRMethodStruct},
language=C]
typedef struct jcr_method_header {
  uint16_t code;
  uint16_t id;
  uint8_t  flags;
  uint8_t  arguments;
  uint8_t  locals;
  uint8_t  stack;
} jcr_method_header_t;
\end{lstlisting}

\begin{description}
\item[code] \hfill \\
Offset within JCR file containing first bytecode corresponding to this method.
\item[id] \hfill \\
Method id used to distinguish API native methods.
\item[flags] \hfill \\
General purposes flags. Currently, only bit 1 is used to indicate if it is constructor method.
\item[arguments] \hfill \\
Number of arguments used by this method.
\item[locals] \hfill \\
Number of local variables used for this method.
\item[stack] \hfill \\
Maximum amount of stack used for this method.
\end{description}

In order to perform the translation process from class file format to JCR file format (taking into account following subsections), a Java application has been designed and developed during this project to perform this translation. It is called JCConverter and received as input a *.class file and creates a *.jcr file by using format described above.

\subsection{Bytecode}\label{SS:JVM:JCR:Bytecode}
\cite{Art:JVMSE7} describes a list of instructions interpretable by standard JVM. Every instruction is represented using just one byte and the combination of them make up a bytecode. For instance, taking into account Java code example of \ref{cod:S:JVM:JCR:JavaCodeExample}.

\medskip
\begin{lstlisting}[
caption={Simple Java Code},
captionpos=b,
label={cod:S:JVM:JCR:JavaCodeExample},
language=Java]
void method() {
    int value_1 = 0;
    int value_2 = 2;

    int sum = (value_1 + value_2);
}
\end{lstlisting}
\medskip

It can be translated to following bytecode:

\medskip
\begin{lstlisting}
03 3C 05 3D 1B 1C 60 3E B1
\end{lstlisting}
\medskip

Where every byte represents an instruction as following:

\medskip
\begin{lstlisting}
instruction [03] - 0000: iconst_0
instruction [3C] - 0001: istore_1
instruction [05] - 0002: iconst_2
instruction [3D] - 0003: istore_2
instruction [1B] - 0004: iload_1
instruction [1C] - 0005: iload_2
instruction [60] - 0006: iadd
instruction [3E] - 0007: istore_3
instruction [B1] - 0008: return
\end{lstlisting}
\medskip

Bytecodes usually represents actions to perform in Java operand stack. Standard JVM uses a stack where every cell or slot can be filled by an integer value. Usual devices use 32 or 64 bits in order to represent an integer, but the JVM designed for this project uses an stack where every slot can be filled by 16 bits. It restricts the available Java primitive types that can be used when Java application for microcontroller is being written. Current design only allows byte, short, boolean and char primitive types. This limitation also exists in most of JVM designed for 8- or 16- bit devices.

Behind of limitation described in previous paragraph, this project is a proof of concept and on the main goals is to show that it is possible to create an easy way to develop applications for $\mu$C by using higher level programming languages like Java, but full design of JVM is out of this scope, then, full support of bytecodes is not currently implemented. Maybe it is a good work for future projects, where could it be interesting to analyze the best way to optimize a JVM or how fully integrate and Standard Java into small devices. \ref{Appx:Bytecodes} described all bytescodes supported by JVM designed for this project.

Due to this bytecode restriction, JCR translation also implies some bytecode translation as described in Table \ref{tab:SS:JVM:JCR:TransBytecodes}

\begin{table}[!htb]
\centering
\begin{tabular}{|c|c|}
\hline 
Class Bytecode & JCR Bytecode \\ 
\hline 
ASTORE & ISTORE \\ 
\hline 
ASTORE\_ X & ISTORE\_ X \\ 
\hline 
ALOAD\_ X & ILOAD\_ X \\ 
\hline 
ACONST\_ NULL & ICONST\_ 0 \\ 
\hline 
IFNULL & IFEQ \\ 
\hline 
IFNONULL & IFNE \\ 
\hline 
I2B & NOP \\ 
\hline 
I2C & NOP \\ 
\hline 
I2S & NOP \\ 
\hline 
\end{tabular}
\caption{Translated bytecodes}
\label{tab:SS:JVM:JCR:TransBytecodes}
\end{table}

\subsection{Calling Native API methods}\label{SS:JVM:JCR:NativeAPI}
Java allows us a mechanism to develop an object oriented application an run it on a microcontroller, but how can it get access to device resources? It is done through API described in \ref{S:JVM:API}.

Java method can be called by other ones or maybe it could require some call to any function in code that make it up. This calling process is usually translated to standard bytecodes INVOKEVIRTUAL, INVOKESPECIAL or INVOKESTATIC followed by two bytes indicating method identifier as shown in example \ref{cod:S:JVM:JCR:JavaCodeExampleMethod}, where \textit{main()} code is translated to bytecode:

\medskip
\begin{lstlisting}
B8 00 0F B1

instruction [B8] - 0000: invokestatic 15
instruction [B1] - 0003: return
\end{lstlisting}
\medskip

Where, 0x000F is the identifier of \textit{method()} in class file.

\medskip
\begin{lstlisting}[
caption={Simple Java Code showing call to one method},
captionpos=b,
label={cod:S:JVM:JCR:JavaCodeExampleMethod},
language=Java]
public static void main() {
    method();
}

public static void method() {
    int value_1 = 0;
    int value_2 = 2;

    int sum = (value_1 + value_2);
}
\end{lstlisting}
\medskip

In case that application requires to use API method, the same mechanism is used but identifier of method is marked as native by setting to '1' the MSB\nomenclature{MSB}{Most Significant Bit} of method identifier. Then, when JVM is going to interpret INVOKEVIRTUAL, INVOKESPECIAL or INVOKESTATIC bytecode, it checks if first bit of method identifier indicates native or not (is set to '1' or '0'). If method identifier indicates native, then it check corresponding identifier of native method and execute corresponding function implemented in native or C, in this case. It allows to get access to $\mu$C resources by using Java API.

\begin{table}[!htb]
\centering
\begin{tabular}{|c|c|c|c|c|c|c|c|c|}
\hline 
Bit 8 & Bit 7 & Bit 6 & Bit 5 & Bit 4 & Bit 3 & Bit 2 & Bit 1 & Meaning \\ 
\hline 
0 & - & - & - & - & - & - & - & Native \\ 
\hline 
1 & - & - & - & - & - & - & - & No Native \\ 
\hline 
- & X & X & X & X & X & X & X & Identifier \\ 
\hline 
\end{tabular}
\caption{Method identifier bit interpretation of first byte}
\label{tab:SS:JVM:JCR:TransBytecodes}
\end{table}

\begin{table}[!htb]
\centering
\begin{tabular}{|c|c|c|c|c|c|c|c|c|}
\hline 
Bit 8 & Bit 7 & Bit 6 & Bit 5 & Bit 4 & Bit 3 & Bit 2 & Bit 1 & Meaning \\ 
\hline 
X & X & X & X & X & X & X & X & Identifier \\ 
\hline 
\end{tabular}
\caption{Method identifier bit interpretation of second byte}
\label{tab:SS:JVM:JCR:TransBytecodes}
\end{table}

\section{Loader}\label{S:JVM:Loader}
Once Java standard application is compiled, generating a \textit{*.class} file, and it is translated to JCR file using JCConverter tool also developed for this project purposes, generating \textit{*.jcr} file, then, full content of this final generated file has to be loaded into JVM in order to start bytecode execution.

Previosuly loaded OS contains both JVM and Loader modules as described in \ref{fig:C:JVM:OverviewDiagram}. Loader is the responsible to manage the loading of external compiled Java applications to the microcontroller OS. If no Java application is loaded and running on $\mu$C, Loader is waiting to receive a Load command through Serial Port as defined in \ref{tab:SS:JVM:Loader:Commands}.

Current unique interface supported to download or remove applications is Serial Port. Some commands and a protocol have been designed for this communication between external entity and Loader listening on $\mu$C.

\subsection{Commands}\label{SS:JVM:Loader:Commands}
Loader is able to process some commands formatted as following structure:

\begin{center}
INS (1 byte) + Data Length (2 bytes) + Data (N Bytes)
\end{center}

Currently supported commands defined for this project are defined in table \ref{tab:SS:JVM:Loader:Commands}.

\begin{table}[!htb]
\centering
\begin{tabular}{|c|c|c|c|}
\hline 
Command & INS & Length & Data \\ 
\hline 
Load & 0x10 & Variable & JCR file content \\ 
\hline 
Delete & 0x11 & 0x0000 & Empty \\ 
\hline 
\end{tabular}
\caption{Loader supported commands}
\label{tab:SS:JVM:Loader:Commands}
\end{table}

\begin{description}
\item[Load] \hfill \\
When OS is running but no Java application is loaded on it, Loader and JVM is waiting that some application is loaded on it in order to start the execution. It is possible to say that OS is empty or waiting. It is in this moment when this command can be sent to Loader in order to load a new Java application. If application is already loaded on JVM, this command is rejected.
\item[Delete] \hfill \\
Once some application is loaded on $\mu$C, it can be removed in order to load a new one by using this command.
\end{description}

\subsection{Protocol}\label{SS:JVM:Loader:Protocol}
In order to send commands specified in \ref{SS:JVM:Loader:Commands}. Following protocol has been designed for this project purposes.

SCB\nomenclature{SCB}{Start Communication Byte} is the first byte sent to the Loader in order to notify that command is going to be sent. SCB is always 0x00. If Loader is not ready to receive commands a 0x91 is returned as a response, otherwise ACK\nomenclature{ACK}{Acknowledgement} with value 0x61 is returned.

After that, INS\nomenclature{INS}{Instruction} indicates the command that is going to be sent. Loader has to answer using the same INS byte to indicate that it support this command. If it does not support this command, 0x92 byte has to be sent as a response.
Once INS has been successfully received, Loader is ready to receive data length and data bytes. Every length and data byte is answered by an ACK with value 0x61 if it has been processed successfully. Otherwise, 0x93 is returned if some error happens during transmission or processing. Loader must answer 0x80 to the last data byte indicating that command has been successfully processed. The full process to send a command to Loader is shown in \ref{fig:SS:JVM:Loader:ProtocolSeq}.

\begin{figure}[H]
\centering
\includegraphics[scale=0.5]{LoaderProtocol.eps}
\caption{Specific project designed protocol sequence to send command to Loader}
\label{fig:SS:JVM:Loader:ProtocolSeq}
\end{figure}

\subsection{Sender}\label{SS:JVM:Loader:Sender}
In order to send JCR file generated as described in \ref{S:JVM:JCR} using Protocol described in \ref{SS:JVM:Loader:Protocol}, a standard Java application has been designed an developed.

It uses Java serial port communication library JSSC (Java Simple Serial Connector) v2.6.0 in order to connect to the Loader through Serial Port using parameters described in Table \ref{tab:SS:JVM:Loader:Sender:Params}

\begin{table}[!htb]
\centering
\begin{tabular}{|c|c|}
\hline 
Parameter & Value \\ 
\hline 
BaudRate & 2400 \\ 
\hline 
Data Bits & 8 \\ 
\hline 
Stop Bits & 1 \\ 
\hline 
Parity & 0 \\ 
\hline 
RTS & enabled \\ 
\hline 
DTS & enabled \\ 
\hline 
\end{tabular}
\caption{Serial Port parameters used by Sender application}
\label{tab:SS:JVM:Loader:Sender:Params}
\end{table}

In order to test o simulate it using Proteus, a virtual serial port is created an terminal is used to send and receive data. VSPE (Virtual Serial Port Emulator) and RealTerm have been used for this purposes, respectively.

\section{JVM}\label{S:JVM:RunTime}
Once application is loaded, then it is time to start JVM execution. JCR data downloaded is interpreted as described in \ref{S:JVM:JCR} and \textit{main()} bytecode is started to be executed as described in \ref{fig:JVMRunTime}.

\begin{figure}[H]
\centering
\includegraphics[scale=0.75]{JVMRunTime.eps}
\caption{JVM basic Run Time}
\label{fig:JVMRunTime}
\end{figure}

Standard JVM uses several Run Time areas during execution of some program as defined in \cite{Art:JVMSE7}. The JVM designed for this project contains 4 areas: Global, Heap, Local and Stack. All of them are dynamically managed by using just one array of unsigned shorts. Figure \ref{fig:JVMRunTimeAreas} shows how these areas are distributed in the unique array used for this purpose.

\begin{figure}[H]
\centering
\includegraphics[scale=0.75]{JVMRunTimeAreas.eps}
\caption{JVM basic Run Time areas distribution in memory array}
\label{fig:JVMRunTimeAreas}
\end{figure}

\begin{description}
\item[Global] \hfill \\
Contains the values of global variables. The corresponding slot identifies the constant. The size is constant during execution of the program.
\item[Heap] \hfill \\
Contains the objects created during application execution, i.e. arrays, Java objects, etc. Size is variable and garbage collection is executed if required space for a new object is required. If there is no more space, an internal error is thrown.
\item[Local] \hfill \\
Contains the values of local constants being used during execution of some specific method. Its size is variable depending on method being executed.
\item[Stack] \hfill \\
Last-in-first-out (LIFO) stack known as its operand stack. It is used to execute the programming operation pushing values an getting results. Its size is variable depending on method being executed.
\end{description}

\subsection{Interruptions}\label{SS:JVM:RunTime:Int}
Diagram described in \ref{fig:JVMRunTime} only described the main loop of JVM where system is interpreting every bytecode of the code but, as defined in \ref{SS:JVM:API:Main}, it is possible to register the application to some possible events. These events are basically interruptions in microcontroller an they have to be handle during execution of JVM if application is registered to them. Diagram shown in \ref{fig:JVMRunTimeWithInt} shows how JVM Run Time manages this interruptions.

\begin{figure}[H]
\centering
\includegraphics[scale=0.75]{JVMRunTimeWithInt.eps}
\caption{JVM basic Run Time}
\label{fig:JVMRunTimeWithInt}
\end{figure}

When Java application calls to the native method \textit{setEvent()}, it is basically managing required microcontroller registers to wait until interruption. When this event or interruption happens, \textit{onEvent()} method is suddenly called, stopping current processing of the application and returning to the same point once \textit{onEvent()} method has been fully processed. For instance, example \ref{cod:SS:JVM:RunTime:Int:JavaCodeExample} shows a code where \textit{main()} (starting application method) only registers the application to EVENT\_ INTERRUPT\_ 0 (INT0 interruption in case of PIC microcontroller) and, after that, it enters in a endless loop. During this final \textit{while}, JVM is executing bytecodes corresponding to the loop itself but it is already registered to the interruption. If INT0 interruption occurs, as $\mu$C is register to it, native handler interruption method is thrown in the middle of JVM execution (\textit{interrupt} in C code). Then a flag indicating that interruption was thrown but not processed is stored and $\mu$C interruption cleared. Then, JVM continues processing bytecode that it was processing before interruption and, at the en of bytecode processing, the previous set flag is checked. As flag is set to '1', then, current status is stored and \textit{onEvent()} is called. After execute this method, JVM returns to previous state and continue processing, i.e. final \textit{while} loop.

\medskip
\begin{lstlisting}[
caption={Interruption Java Code example},
captionpos=b,
label={cod:SS:JVM:RunTime:Int:JavaCodeExample},
language=Java]
public class Main extends MicroApplication implements
  MicroApplicationListener {

    public void main() {
        SetEvent(PortConstants.EVENT_INTERRUPT_0);

        while (true) {}
    }

    public void onEvent(byte event) {
        if (PortConstants.EVENT_INTERRUPT_0 == event) {
            // Code...
        }
    }

}
\end{lstlisting}
\medskip

It means that, when interruption really occurs, it is not exactly triggered at the moment, it is processed with a little bit delay. JVM is, in general terms, a very big loop processing every bytecode. At the end of every loop check if some registered interruption is pending to be processed by \textit{onEvent()} method. It is important to keep in mind that, if application is registered to some event and \textit{main()} code has been fully processed, JVM still keep in a internal loop waiting to any interruption. It means that example \ref{cod:SS:JVM:RunTime:Int:JavaCodeExample} is the same as example \ref{cod:SS:JVM:RunTime:Int:JavaCodeExampleNoLoop}.

\medskip
\begin{lstlisting}[
caption={Interruption Java Code example without final loop},
captionpos=b,
label={cod:SS:JVM:RunTime:Int:JavaCodeExampleNoLoop},
language=Java]
public class Main extends MicroApplication implements
  MicroApplicationListener {

    public void main() {
        SetEvent(PortConstants.EVENT_INTERRUPT_0);
    }

    public void onEvent(byte event) {
        if (PortConstants.EVENT_INTERRUPT_0 == event) {
            // Code...
        }
    }

}
\end{lstlisting}
\medskip

\section{Microcontroller adaptation}\label{SS:JVM:Micros}
Another objective of this project is to create an interopeable JVM that works for every device. It is possible, but some specific parts of every $\mu$C are different and cannot be common. For instance, one microcontroller can have Timer 0 and Timer 1 available but another one maybe just have Timer 0. Then, if a Java application tries to get this resource by using API designed in this project with following code:

\medskip
\begin{lstlisting}[language=Java]
Timer timer = Timer.getTimer(TimerConstants.TIMER1);
\end{lstlisting}
\medskip

First device must return it, but second one cannot and should return a \textit{null} object.

Other scenarios similar to the one specified above problem can happen internally in OS developed. Basically, the main problems is how to write or read some data or memory from every specific $\mu$C. These mechanisms are specified in datasheet of every microcontroller and the can be totally different every time. For that reason, the OS has been designed containing several interfaces separating common part that can be used for all the devices from device specific component as shown in \ref{fig:uCIntfs}.

\begin{figure}[H]
\centering
\includegraphics[scale=0.5]{MicrocontrollerAdaptation.eps}
\caption{Internal interface speration in microcontroller system designed for this project}
\label{fig:uCIntfs}
\end{figure}

It makes easy to adapt system to several devices. Only $\mu$C specific part has to be developed and integrated on system, but rest o functionality is always the same.
This design also means that every $\mu$C should have its own compilation, but the same Java application will work on every compilation, i.e. final Java application should never change.

\subsection{Memory Management}\label{SS:JVM:Micros:MM}
\ref{fig:uCIntfs} shows that exists an internal interface that separated how to read or write data in one microcontroller from main system. It allows that following code get the same value regardless $\mu$C used in this  compilation. Main system should not know how to write or read a byte in memory.

\medskip
\begin{lstlisting}[
caption={Reading byte using internal Memory Management interface},
captionpos=b,
label={cod:SS:JVM:Micros:MM:Reading},
language=C]
uint8_t bytecode = Mm_GetU08(pc);
\end{lstlisting}
\medskip

Interface designed for this project only contains methods like described in \ref{cod:SS:JVM:Micros:MM:Interface}. It is applied for signed types or arrays too.

\medskip
\begin{lstlisting}[
caption={Internal Memory Management interface},
captionpos=b,
label={cod:SS:JVM:Micros:MM:Interface},
language=C]

/**
 * Reads a byte from specified address.
 *
 * @param address Address where byte to read is placed.
 * @return Byte read.
 */
uint8_t  Mm_GetU08(mm_address_t address);

/**
 * Reads a unsigned short from specified address.
 *
 * @param address Address where bytes to read are placed.
 * @return Unsigned short read.
 */
uint16_t Mm_GetU16(mm_address_t address);

/**
 * Writes a byte on specified address.
 *
 * @param address Address where byte is going to be placed.
 */
void Mm_SetU08(mm_address_t address, uint8_t value);

/**
 * Writes a unsigned short on specified address.
 *
 * @param address Address where bytes are going to be placed.
 */
void Mm_SetU16(mm_address_t address, uint16_t value);
\end{lstlisting}
\medskip

For instance, PIC18F4520 specifies different steps to follow if user needs to write or read from Flash, EEPROM\nomenclature{EEPROM}{Electrically Erasable Programmable Read-Only Memory}or RAM memory. In order to read and write program memory, there are two operations that allow the PIC18F4520 processor to move bytes between the program memory space and the data RAM: Table Read (TBLRD) and Table Write (TBLWT). Table reads and table writes move data between these two memory spaces through an 8-bit register (TABLAT). Table read operations retrieve data from program memory and places it into the data RAM space. Table write operations store data from the data memory space into holding registers in program memory. The procedure to write and read are specified in detail in PIC18F4520 data sheet. The EEADR register is used to address the data EEPROM for read and write operations. The 8-bit range of the register can address a memory range of 256 bytes (0x00 to 0xFF).

%\subsubsection{PIC18F4520}\label{SSS:JVM:Micros:MM:PIC18F4520}
%\subsubsection{PIC16}\label{SSS:JVM:Micros:MM:PIC16}
%\subsubsection{AVR}\label{SSS:JVM:Micros:MM:AVR}

Then, every new device added has to develop its following read or write mechanism in  methods described in \ref{cod:SS:JVM:Micros:MM:InterfaceDev}. Then, methods of interface \ref{cod:SS:JVM:Micros:MM:Interface} must call to device interface methods to write and read, getting in that way a division of common and device-specific parts.

\medskip
\begin{lstlisting}[
caption={Device Memory Management interface},
captionpos=b,
label={cod:SS:JVM:Micros:MM:InterfaceDev},
language=C]
/**
 * Reads bytes starting on specified address of memory.
 *
 * @param address Address where first bytes to read is
 * placed.
 * @param bytes Number of bytes to read.
 * @param data Pointer to array where bytes read are
 * going to be copied.
 */
void Mm_ReadNVM(mm_address_t address, mm_address_t bytes, uint8_t *data);

/**
 * Writes bytes starting on specified address of memory.
 *
 * @param address Address where first bytes to write is
 * going to be placed.
 * @param bytes Number of bytes to write.
 * @param data Pointer to byte array containing data to
 * write.
 */
void Mm_WriteNVM(mm_address_t address, mm_address_t bytes, uint8_t *data);
\end{lstlisting}
\medskip

\subsection{API}\label{SS:JVM:Micros:API}
Every microcontroller can have a different resources and a different way to configure it. Taking as an example the Java package related to pin modifications, when Java application call to method \textit{setPinToOne()}, every $\mu$C has its own register for this purposes. Then, as defined in \ref{SS:JVM:JCR:NativeAPI}, when JVM is executing bytecodes indicating that is required to execute the native method \textit{setPinToOne()}, the system will call to Native API interface corresponding method defined as follows:

\medskip
\begin{lstlisting}[language=C]
void Api_PortRegistry_SetPinToOne(uint8_t pin);
\end{lstlisting}
\medskip

This method belongs to Native API interface as explained in \ref{fig:uCIntfs}. Then, this method should call to device-specific methods which have always the same header but definition will depend on device compilation, doing a separation between resources an API native implementations. It means that, for every new device, following method should be implemented:

\medskip
\begin{lstlisting}[language=C]
void Port_SetPin(uint8_t pins, uint8_t type)
\end{lstlisting}
\medskip

Because it is going the function called by \textit{Api\_ PortRegistry\_ SetPinToOne} method. It is applied for all native methods implemented in API specified in \ref{S:JVM:API}.

\section{CodeStructure}\label{S:JVM:Code}
Organization of code is explained in \ref{Appx:CodeStructure}

%\section{Example}\label{S:JVM:Example}

\chapter{Direct interface circuit for resistive sensors}\label{C:Res}
\section{Theoretical Circuit Analysis}\label{S:Res:Analysis}
As direct resistive sensor-to-microcontroller interfaces are basically based on measuring discharging time of an RC\nomenclature{RC}{Resistor-Capacitor} circuits, first step before to analyze complete circuit is to study, remember or briefly introduce its operating principle. The RC\nomenclature{R}{Resistor}\nomenclature{C}{Capacitor} output relation between voltage (V\nomenclature{V}{Voltage}) and time (t\nomenclature{t}{Time}) is shown in Figure \ref{fig:RCChargeDischargeGraphic}.

\begin{figure}[h]
\centering
\includegraphics[scale=0.75]{RCChargeDischargeGraphic.eps}
\caption{Charging and discharging output of RC circuit}
\label{fig:RCChargeDischargeGraphic}
\end{figure}

Capacitor is initially charged until maximum voltage threshold $V_{TH}$. Until arrive to this state, the graphic can be modeled as \eqref{eq:ChargeRC} and the time to reach maximum threshold value can be determined by \eqref{eq:ChargeTime}.

\begin{equation}
\label{eq:ChargeRC}
v_{o}(t) = V_{MAX}(1-e^{-\frac{t}{RC}})
\end{equation}

\begin{equation}
\label{eq:ChargeTime}
T_{c} = RC\ln(\frac{V_{MAX}}{V_{MAX}-V_{TH}})
\end{equation}

Nevertheless, capacitor discharge and the time required to get $V_{TL}$ can be obtained by \eqref{eq:ChargeRC} and \eqref{eq:DischargeTime}, respectively.

\begin{equation}
\label{eq:DischargeRC}
v_{o}(t) = V_{MAX}e^{-\frac{t}{RC}}
\end{equation}

\begin{equation}
\label{eq:DischargeTime}
T_{d} = RC\ln(\frac{V_{MAX}}{V_{TL}})
\end{equation}

Taken into account the internal resistance of the microcontroller pin $R_{n}$, the $R$ from \eqref{eq:DischargeTime} can be expressed as $R=(R+R_{n})$. If factor $k_{R}$ is defined as $k_{R}=C\ln(\frac{V_{MAX}}{V_{TL}})$, equation \eqref{eq:DischargeTime} can be redefined as \eqref{eq:DischargeTimeRedefined}.

\begin{equation}
\label{eq:DischargeTimeRedefined}
N = k_{R}(R+R_{n})
\end{equation}

Where $N$ can be defined as number of counts obtained by microcontroller during capacitor discharge.
\medskip

Taking below definitions, we could measure discharging time of an unknown resistor or resistive sensor $R_{x}$ and, knowing $R_{n}$, $C$, $V_{MAX}$ and $V_{TL}$, $R_{x}$ should be also known, but all of those parameters are usually constrained by external factors and it results on imprecise final values. In order to avoid it, some calibration techniques use several reference parameters to alleviate those possible variations. Calibration techniques are basically distinguished between them by number of reference components used to obtain most accuracy and resolution. Figures \ref{fig:MicrocontrollerBasedInterfaceCircuitResistor2Points} and \ref{fig:MicrocontrollerBasedInterfaceCircuitResistor3Points} represents two and three points calibration circuit based on resistive sensor, respectively. Both systems are analized during this document. Single-point calibration technique is out of scope of this report.
\medskip

Two-point calibration technique only uses two calibrator resistors $R_{c1}$ and $R_{c2}$ and three measurements are performed:
\medskip

\begin{itemize}
\item Number of counts $N_{x}$ during capacitor $C$ discharge by unknown resistor or sensor $R_{x}$.
\item Number of counts $N_{c1}$ during capacitor $C$ discharge by calibration resistor $R_{c1}$.
\item Number of counts $N_{c2}$ during capacitor $C$ discharge by calibration resistor $R_{c2}$.
\end{itemize}
\medskip

Then $R_{x}$ can be computed by \eqref{eq:RxTwoPoints}.

\begin{equation}
\label{eq:RxTwoPoints}
R_{x} = \frac{N_{x}-N_{c1}}{N_{c2}-N_{c1}}(R_{c2}-R_{c1})+R_{c1}
\end{equation}

Three-point calibration technique may be considered a variation of two-point calibration technique where $R_{c1}$ is considered zero and an additional resistor $R_{o}$ is placed to improve exponential discharge of capacitor. Then $R_{x}$ can be computed by \eqref{eq:RxThreePoints}.

\begin{equation}
\label{eq:RxThreePoints}
R_{x} = \frac{N_{x}-N_{c1}}{N_{c2}-N_{c1}}R_{c2}
\end{equation}

\begin{figure}[h]
\centering
	\begin{subfigure}{0.45\textwidth}
	\includegraphics[width=\textwidth]{MicrocontrollerBasedInterfaceCircuitResistor2Points.eps}
	\caption{Two calibration points}
	\label{fig:MicrocontrollerBasedInterfaceCircuitResistor2Points}
	\end{subfigure}
	~
	\begin{subfigure}{0.45\textwidth}
	\includegraphics[width=\textwidth]{MicrocontrollerBasedInterfaceCircuitResistor3Points.eps}
	\caption{Three calibration points}
	\label{fig:MicrocontrollerBasedInterfaceCircuitResistor3Points}
	\end{subfigure}
\caption{Microcontroller-based interface circuit for a resistive sensors}
\end{figure}

Figures \ref{fig:Calibration2Points} and \ref{fig:Calibration3Points} show how nonlinearities are correct depending on two points or three points calibration technique, respectively, applying circuits from Figures \ref{fig:MicrocontrollerBasedInterfaceCircuitResistor2Points} and \ref{fig:MicrocontrollerBasedInterfaceCircuitResistor3Points}.
\medskip

\begin{figure}[!ht]
\centering
	\begin{subfigure}{0.75\textwidth}
	\includegraphics[width=\textwidth]{Calibration2Points.eps}
	\caption{Two calibration points}
	\label{fig:Calibration2Points}
	\end{subfigure}
	~
	\begin{subfigure}{0.75\textwidth}
	\includegraphics[width=\textwidth]{Calibration3Points.eps}
	\caption{Three calibration points}
	\label{fig:Calibration3Points}
	\end{subfigure}
\caption{Relation between $N_{x}$ and $R_{x}$ applied to}
\end{figure}

But, all of previous analysis does not take into account internal resistance from microcontroller $R_{n}$. \cite{Art:Accuracy} analyses equations \eqref{eq:RxTwoPoints} and \eqref{eq:RxThreePoints} including internal microcontroller resistance resulting equation \eqref{eq:RxInternalRes}.

\begin{equation}
\label{eq:RxInternalRes}
R_{x}^{*} = \frac{R_{c2}-R_{c1}}{R_{c2}-R_{c1}+\Delta R_{12}}R_{x}+\frac{R_{c1}\Delta R_{13}-R_{c2}\Delta R_{23}}{R_{c2}-R_{c1}+\Delta R_{12}}
\end{equation}

Where $\Delta R_{23}=R_{n,2}-R_{n,3}$, $\Delta R_{13}=R_{n,1}-R_{n,3}$ and $\Delta R_{12}=R_{n,1}-R_{n,2}$. Then, relative error can be expressed as equation \eqref{eq:RelativeErrorInternalRes}.

\begin{equation}
\label{eq:RelativeErrorInternalRes}
e_{r} = \left|\frac{-\Delta R_{12}}{R_{c2}-R_{c1}+\Delta R_{12}}+\frac{1}{R_{x}}\frac{R_{c1}\Delta R_{13}-R_{c2}\Delta R_{23}}{R_{c2}-R_{c1}+\Delta R_{12}}\right|
\end{equation}

\section{Code Analysis}\label{S:Res:Code}
\subsection{Finite State Machine}\label{S:Res:Code:FSM}
Finite State Machine (FSM\nomenclature{FSM}{Finite State Machine}) is a computational model, usually used by hardware or some types of software, where the aplication, program or machine can have one or more different states and it will be always in one of them. This state can be changed depending on some events which are represented as an entrance on this state. Those events may, or not, generate an output and, accordingly, a new state transition.
\medskip

Figure \ref{fig:FSM} shows an easy example to represent the FSM concept. The application represented in Figure \ref{fig:FSM} can remain in three possible states while it is being executed: $S1$, $S2$ and $S3$. If application stays on $S1$, the event $E_{2}$ would imply a transition to state $S2$ but event $E_{1}$ would change the application state to $S3$. In that way, Figure \ref{fig:FSM} also shows what transitions are available and how to perform them.

\begin{figure}[h]
\centering
	\begin{subfigure}[b]{0.45\textwidth}
	\includegraphics[width=\textwidth]{FSM.eps}
	\caption{Finite State Machine}
	\label{fig:FSM}
	\end{subfigure}
	~
	\begin{subfigure}[b]{0.25\textwidth}
	\includegraphics[width=\textwidth]{Sequential.eps}
	\caption{Sequential}
	\label{fig:Sequential}
	\end{subfigure}
\caption{Programming diagrams}
\end{figure}

This concept it is easily applicable to the PIC software programmed for this report. Those applications has been based on previous programs based on sequential programming, where the application has no possibilities to move to several states depending on different events, in that case, only one transition is available as shown in Figure \ref{fig:Sequential}.
\subsection{Timers and Capture Mode}\label{S:Res:Code:Timers}
For this project, a Microchip PIC18F4520 has been used. It provides 4 Timers and a CCP\nomenclature{CCP}{Capture Compare PWM}  module. In order to count the time spend to discharge a capacitor, Timers 1 and 2 have been used for this purposes. when Timer 1 is used, it is configured as 16 bits and prescale as 1:1. If Timer 2 is set, it is also configured as 16 bits, prescale as 1:1 (8 bits configuration is not possible for Timer 2) and postcale also is 1:1. Prescale and postcale 1:1 means that counter increments every clock cycle. Timer overflow interruption is not used because timers values are going to be read once, INT0 or CCP1 interruption is thrown.

In order to detect capacitor discharge, two methods have been used during this project. First one is interruption by flag detection in port defined as INT0 and second one is the Capture Mode provided by this PIC. Capture mode introduces more accuracy during time counting because, when INT0 is used, interruption is thrown and Timer value should be requested by software, but Capture Mode stores directly the value of timer when interruption occurs. It allows to read Timer value avoiding possible undesired counts during timer reading processing.
\subsection{Analysis}\label{SS:Res:Code:Analysis}
As explained during introduction of this report, PIC18F4520 microcontroller has been used to analyze circuits from Figures \ref{fig:MicrocontrollerBasedInterfaceCircuitResistor2Points} and \ref{fig:MicrocontrollerBasedInterfaceCircuitResistor3Points}. As explained during Section \ref{S:Res:Analysis}, they are based on charge and discharge of RC circuit.
\medskip

Taken into account a RC circuit connected to a microcontroller pin, following states can be configured on this pin in order to either charge, discharge the capacitor $C$ or set as high-impedance input (HZ\nomenclature{HZ}{High-Impedance}). Those states can be set up modifying TRIS and PORT registers of corresponding pin port bits as defined in Table \ref{tab:PinConfigurations}.
\medskip

\begin{table}[h]
\centering
\begin{tabular}{|c|c|c|}
\hline
Pin Configuration & TRIS bit register value & PORT bit register value \\
\hline
Charge & 0 & 1 \\
\hline
Discharge & 0 & 0 \\
\hline
High-Impedance & 1 & - \\
\hline
\end{tabular}
\caption{Microcontroller pin configuration}
\label{tab:PinConfigurations}
\end{table}
\medskip

In Figures \ref{fig:MicrocontrollerBasedInterfaceCircuitResistor2Points} and \ref{fig:MicrocontrollerBasedInterfaceCircuitResistor3Points}, microcontroller pins connected to $R_{x}$, $R_{c1}$ and $R_{c2}$ should be configured as defined in Table \ref{tab:DischargePinConfigurationsRes} during capacitor discharge. Capacitor charge is performed configuring Pin 4 as Charge and Pin 1, Pin 2 and Pin 3 as High-Impedance.
\medskip

\begin{table}[h]
\centering
\begin{tabular}{|c|c|c|c|}
\hline
$N$ result & Pin 1 & Pin 2 & Pin 3 \\
\hline
$N_{x}$ & HZ & HZ & Discharge \\
\hline
$N_{c1}$ & HZ & Discharge & HZ \\
\hline
$N_{c2}$ & Discharge & HZ & HZ \\
\hline
\end{tabular}
\caption{Pin configurations during discharging time on resistive sensor system}
\label{tab:DischargePinConfigurationsRes}
\end{table}
\medskip

Pin 4 is also set to HZ during discharging process and it is configured as interruption when rising edge is detected on its port pin. It is configured on rising edge instead of falling edge (capacitor discharge) because an external Schmitt trigger is connected to Pin 4, then signal is inverted.
\medskip

Figure \ref{fig:ResistorsCodeDiagram} shows a diagram explaining the programming code of application loaded into PIC18F5420 microcontroller.
\medskip

\begin{figure}[!ht]
\centering
\includegraphics[scale=0.5]{ResistorsCodeDiagram.eps}
\caption{Code diagram of application to measure resistor sensor}
\label{fig:ResistorsCodeDiagram}
\end{figure}

\colorbox{yellow}{Añadir explicación detallada del diagrama de bloques!!}
\subsection{Native}\label{SS:Res:Code:Native}
\subsection{Java}\label{SS:Res:Code:Java}
\section{Implemented Circuit}\label{S:Res:Circuit}
Following concepts described in \ref{S:Res:Analysis} and \ref{S:Res:Code}, Figures \ref{fig:uR2PointsReal}, \ref{fig:uR3PointsReal}, \ref{fig:uR2PointsCCPReal} and \ref{fig:uR3PointsCCPReal} show the final circuit studied during this first part of this report.
\medskip

\begin{figure}[!ht]
\centering
	\begin{subfigure}{0.75\textwidth}
	\includegraphics[width=\textwidth]{MicrocontrollerBasedInterfaceCircuitResistor2PointsReal.eps}
	\caption{Two calibration points}
	\label{fig:uR2PointsReal}
	\end{subfigure}
	~
	\begin{subfigure}{0.85\textwidth}
	\centering
	\includegraphics[width=\textwidth]{MicrocontrollerBasedInterfaceCircuitResistor3PointsReal.eps}
	\caption{Three calibration points}
	\label{fig:uR3PointsReal}
	\end{subfigure}
\caption{Final result circuits for resistor sensors for Time capture mode}
\label{fig:FinalCircuits}
\end{figure}

\begin{figure}[!ht]
\centering
	\begin{subfigure}{0.75\textwidth}
	\centering
	\includegraphics[width=\textwidth]{MicrocontrollerBasedInterfaceCircuitResistor2PointsCCPReal.eps}
	\caption{Two calibration points}
	\label{fig:uR2PointsCCPReal}
	\end{subfigure}
	~
	\begin{subfigure}{0.85\textwidth}
	\centering
	\includegraphics[width=\textwidth]{MicrocontrollerBasedInterfaceCircuitResistor3PointsCCPReal.eps}
	\caption{Three calibration points}
	\label{fig:uR3PointsCCPReal}
	\end{subfigure}
\caption{Final result circuits for resistor sensors for CCP configuration}
\label{fig:FinalCircuitsCCP}
\end{figure}

Those circuits have been implemented using a PICDEM 2 Plus evaluation board and external breadboard. PICDEM 2 Plus has 40-pin DIP\nomenclature{DIP}{Dual In-Line Package} socket, which corresponds to PIC18F4520 (microcontroller used during this experiments). It also contains an RS-232 interface, which has been used to transmit the digital counters $N_{x}$, $N_{c1}$ and $N_{c2}$ obtained by application described in \ref{S:Res:Code}. The evaluation board has been supplied by 9V DC power supply charger because it provides a +5V regulator for direct input from 9V, 100 mA AC/DC wall adapter.
\medskip

PICDEM 2 Plus provides reset button and on-board external oscillator for 4 MHz crystal oscillator. A 4 MHz FOX F1100E oscillator has been used. As clock preescale 1:1 has been configured for the application, the counting time-base ($T_{s}$) used to get digital counts is 1 $\mu$s.
\medskip

A CD40106BE CMOS has been used as external Schmitt trigger. 4 MKT capacitors of 0.22 $\mu$F, 1 $\mu$F, 2.2 $\mu$F and 4.7 $\mu$F have been used to as $C$ in order to analyze the final result depending on capacitor value used. Following range of resistance values has been used as $R_{x}$: 825 $\Omega$, 866 $\Omega$, 909 $\Omega$, 1000 $\Omega$, 1210 $\Omega$, 1330 $\Omega$, 1400 $\Omega$, 1472 $\Omega$, 1540 $\Omega$. The real values measured for those $R_{x}$ used are: 826 $\Omega$, 863 $\Omega$, 905 $\Omega$, 1000 $\Omega$, 1208 $\Omega$, 1329 $\Omega$, 1398 $\Omega$, 1470 $\Omega$, 1539 $\Omega$. Table \ref{tab:CalibrationResUsedInR} shows values used for calibration resistors. Table \ref{tab:CalibrationResUsedInR} shows values used as calibration resistances. All of them are metallic film resistors and has been measured with ISO-TECH IDM203 bench multimeter.

\begin{savenotes}
	\begin{table}[!ht]
	\centering
		\begin{tabular}{|c|c|c|c|c|c|c|}
		\hline 
		& $R_{c1}$ ($\Omega$) & $R_{c1}$ ($\Omega$) & $R_{c2}$ ($\Omega$) & $R_{c2}$ ($\Omega$) & $R_{0}$ ($\Omega$) & $R_{0}$ ($\Omega$) \\
		\hline
		2 Points & 909 & 906 & 1330 & 1328 & 0 & 0 \\
		\hline
		3 Points & 0 & 0 & 1472 & 1470 & 340 & 339.3 \\
		\hline
		\end{tabular}
	\caption{Theoretical and Real values used for calibration resistances $R_{c1}$, $R_{c2}$ and $R_{0}$}
	\label{tab:CalibrationResUsedInR}
	\end{table}
\end{savenotes}

The values for $R_{c1}$, $R_{c2}$ and $R_{0}$ has been chosen following recommendations explained in \cite{Art:Accuracy}.
\section{Experimental results}\label{S:Res:Results}
\subsection{Native}\label{S:Res:Results:Native}
100 measures has been performed for every combination of capacitor $C$, unknown resistor $R_{x}$ and capture mode. Figures \ref{fig:RelativeErrorResistive2P} and \ref{fig:RelativeErrorResistive3P} show the relative error obtained for those results without take into account microcontroller internal resistance $R_{n}$.

\begin{figure}[!htb]
\centering
	\begin{subfigure}{0.45\textwidth}
	\includegraphics[width=\textwidth]{PlotR2PT1_Error.eps}
	\caption{Using Timer 1}
	\label{fig:PlotR2PT1Error}
	\end{subfigure}
	~
	\begin{subfigure}{0.45\textwidth}
	\includegraphics[width=\textwidth]{PlotR2PT1CCP1_Error.eps}
	\caption{Using CCP}
	\label{fig:PlotR2PT1CCP1Error}
	\end{subfigure}
\caption{Relative Error from experimental results for two points calibration and resistive tests}
\label{fig:RelativeErrorResistive2P}
\end{figure}

\begin{figure}[!htb]
\centering
    \begin{subfigure}{0.45\textwidth}
	\includegraphics[width=\textwidth]{PlotR3PT1_Error.eps}
	\caption{Using Timer 1}
	\label{fig:PlotR3PT1Error}
	\end{subfigure}
	~
	\begin{subfigure}{0.45\textwidth}
	\includegraphics[width=\textwidth]{PlotR3PT1CCP1_Error.eps}
	\caption{Using CCP}
	\label{fig:PlotR3PT1CCP1Error}
	\end{subfigure}
\caption{Relative Error from experimental results for three points calibration and resistive tests}
\label{fig:RelativeErrorResistive3P}
\end{figure}

The relative error oscillates between 0.07\% and 0.26\% if $C$ = 0.22 $\mu$F. For higher values of $C$, this relative error oscillates between 0.07\% and 0.01\%. The error is close to 0 when $R_{x}$ values are close to calibrator resistor $R_{c2}$ in case of two points calibration technique. In case of three points technique, the error is close to 0 when $R_{x}$ values are close to calibrator resistor $R_{c2}$, $R_{c1}$ and a midpoint between them. This behavior corresponds to calibration technique shown on Figures \ref{fig:Calibration2Points} and \ref{fig:Calibration3Points}. Figures \ref{fig:DeviationResistive2P} and \ref{fig:DeviationResistive3P} show the standard deviation obtained for every experiment.
\medskip

\begin{figure}[!htb]
\centering
	\begin{subfigure}{0.45\textwidth}
	\includegraphics[width=\textwidth]{PlotR2PT1_Deviation.eps}
	\caption{Using Timer 1}
	\label{fig:PlotR2PT1Deviation}
	\end{subfigure}
	~
	\begin{subfigure}{0.45\textwidth}
	\includegraphics[width=\textwidth]{PlotR2PT1CCP1_Deviation.eps}
	\caption{Using CCP}
	\label{fig:PlotR2PT1CCP1Deviation}
	\end{subfigure}
\caption{Standard Deviation from experimental results for two points calibration and resistive tests}
\label{fig:DeviationResistive2P}
\end{figure}

\begin{figure}[!htb]
\centering
    \begin{subfigure}{0.45\textwidth}
	\includegraphics[width=\textwidth]{PlotR3PT1_Deviation.eps}
	\caption{Using Timer 1}
	\label{fig:PlotR3PT1Deviation}
	\end{subfigure}
	~
	\begin{subfigure}{0.45\textwidth}
	\includegraphics[width=\textwidth]{PlotR3PT1CCP1_Deviation.eps}
	\caption{Using CCP}
	\label{fig:PlotR3PT1CCP1Deviation}
	\end{subfigure}
\caption{Standard Deviation from experimental results for three points calibration and resistive tests}
\label{fig:DeviationResistive3P}
\end{figure}

If internal resistors $R_{n}$ summarized in \ref{tab:InternalResistors} are taken into account as defined in equations \eqref{eq:RxInternalRes} and \eqref{eq:RelativeErrorInternalRes}. The internal resistors has been computed with PIC18F4520 put in PICDEM 2 Plus.
\medskip

\begin{table}[!htb]
\centering
\begin{tabular}{|c|c|c|}
\hline 
Port Pin & Internal R & Value $\Omega$ \\ 
\hline 
RB1 & $R_{n,2}$ & 19,7591 \\ 
\hline 
RB2 & $R_{n,3}$ & 19,6754 \\ 
\hline 
RB3 & $R_{n,4}$ & 19,5290 \\ 
\hline 
\end{tabular} 
\caption{Internal PIC18F4520 resistor on PICDEM 2 Plus}
\label{tab:InternalResistors}
\end{table}

Figures \ref{fig:PlotR2PT1ErrorRn}, \ref{fig:PlotR2PT1CCP1ErrorRn}, \ref{fig:PlotR3PT1ErrorRn} and \ref{fig:PlotR3PT1CCP1ErrorRn} show the experimental an theoretical results obtained applying $R_{n}$. Those figures only show theoretical and experimental results for $C=2.2$ $\mu$F and $C=4.7$ $\mu$F.
\medskip

\begin{figure}[!htb]
\centering
	\begin{subfigure}{0.45\textwidth}
	\includegraphics[width=\textwidth]{PlotR2PT1_ErrorRn.eps}
	\caption{Using Timer 1}
	\label{fig:PlotR2PT1ErrorRn}
	\end{subfigure}
	~
	\begin{subfigure}{0.45\textwidth}
	\includegraphics[width=\textwidth]{PlotR2PT1CCP1_ErrorRn.eps}
	\caption{Using CCP}
	\label{fig:PlotR2PT1CCP1ErrorRn}
	\end{subfigure}
\caption{Relative Error from experimental results for two points calibration and resistive tests taking into account $R_{n}$}
\label{fig:RelativeErrorRnResistive2P}
\end{figure}

\begin{figure}[!htb]
\centering
    \begin{subfigure}{0.45\textwidth}
	\includegraphics[width=\textwidth]{PlotR3PT1_ErrorRn.eps}
	\caption{Using Timer 1}
	\label{fig:PlotR3PT1ErrorRn}
	\end{subfigure}
	~
	\begin{subfigure}{0.45\textwidth}
	\includegraphics[width=\textwidth]{PlotR3PT1CCP1_ErrorRn.eps}
	\caption{Using CCP}
	\label{fig:PlotR3PT1CCP1ErrorRn}
	\end{subfigure}
\caption{Relative Error from experimental results for three points calibration and resistive tests taking into account $R_{n}$}
\label{fig:RelativeErrorRnResistive3P}
\end{figure}

Experimental results tendency of results obtained for 3 points experiments trend to theoretical line but spreading is higher than 2 points experiments. Lower values on 2 points experiments are more distant from theoretical results in comparison with 3 points results.
\medskip

Figure \ref{fig:PlotAllErrorsRn} shows a graphic representing all capture modes and results for $C=4.7$ $\mu$F in order to represent a comparison between them and theoretical values.

\begin{figure}[!htb]
\centering
\includegraphics[scale=0.5]{PlotAllErrorsRn.eps}
\caption{Relative Error taken into account $R_{n}$ of all capture modes}
\label{fig:PlotAllErrorsRn}
\end{figure}

\subsection{Java}\label{S:Res:Results:Java}
\cleardoublepage
\phantomsection
\chapter{Conclusions}\label{C:Conclusions}
\colorbox{yellow}{TODO}

%%%  BIBLIOGRAFIA
%%%%%%%%%%%%%%%%%%%%%%%%%%%%%%%%%%%%%%%%%%%%%%%%%%%%%%%%%%%%%%%%%%%%%%%%%%

%%% Per la bibliografia hi ha 2 opcions: generarla amb la utilitat BibTeX 
%%%                                      o fer-la ''a ma''
%%% NOTA: podeu trobar facilment informació sobre BibTeX a:
%%%  http://www.ctan.org/tex-archive/biblio/bibtex/contrib/doc/

%%% OPCIO 1: BibTeX (recomanat) -> descomentar les comandes seguents:
%\bibliographystyle{unsrt}   %% Estil de bibliografia EETAC
%\cleardoublepage
%\phantomsection
% Indicar aqui el(s) fitxer(s) que contenen la bibliografia
%\bibliography{fitxer1,...,fitxerN}  
%\pdfbookmark{Bibliografia}{sec:biblio}

%%% OPCIO 2: bibliografia manual
%%%
%%% L'argument d'entrada es el numero de referencies que s'inclouen
\cleardoublepage
\phantomsection
\begin{thebibliography}{2}

%% Llibres:  Autor/s (cognoms i inicials dels noms), títol del llibre (en cursiva), editor, ciutat i any de publicació. Quan es cita el capítol d'un llibre s'ha d'indicar el títol del capítol (entre cometes), el títol del llibre (en cursiva) i els números de pàgines amb la primera i la darrera incloses.

%%  Exemple de capitol en llibre
%\bibitem{prova1} 
%Cognoms-autor, Inicial-nom.
%``Títol del capítol''. {\it Títol del llibre}.
%(Editor. Ciutat. Any publicació): pagina1--paginaN.

\bibitem{Art:JavaForMicro} 
Sérgio Akira Ito, Luigi Carro and Ricardo Pezzulo Jacobi.
''Making Java Work for Microcontroller Applications''. {\it IEEE Xplore}.100--110. (September-October 2001)

\bibitem{Art:Darjeeling} 
Niels Brouwers, Peter Corke and Koen Langendoen.
''Darjeeling, a Java Compatible Virtual Machine for Microcontrollers''. {\it ACM/IFIP/USENIX Middleware '08 Conference Companion}. (December 2008)

\bibitem{Art:NanoVM} 
Ricardo Costa, David Ludovino and Nuno Pinheiro.
''nanoVM on AVR microcontroller: evaluation, extension and deployment''. {\it Group 7 shift AVExe36L03. Departamento de Engenharia Informatica, DEI. Instituto Superior Tecnico, IST. Lisboa, Portugal}. (2006)

\bibitem{Art:JVMSE7} 
Tim Lindhlom, Frank Yellin, Gilad Bracha and Alex Buckley.
''The Java Virtual Machine Specification Java SE 7 Edition''. {\it Oracle}. (February 2013)

\bibitem{Art:Uncertainty} 
Josep Jordana and Ferran Reverter and Ramon Pallàs-Areny.
''Uncertainty in resistence measurements based on microcontrollers with embedded time counter''. {\it IEEE Intrumentation and Measurement Technology Conference}. (2003)

\bibitem{Art:Capacitive} 
Ferran Reverter and Óscar Casas.
''Direct interface circuit for capacitive humidity sensors''. {\it Sensors and Actuators A 143 315-322}. (2008)

\bibitem{Art:Accuracy} 
Ferran Reverter and Josep Jordana and Manel Gasulla and Ramon Pallàs-Areny.
''Accuracy and resolution of direct resistive sensor-to-microcontroller interfaces''. {\it Sensors and Actuators A 121}. (2005)

\bibitem{Art:IntfCircuits} 
Ferran Reverter and Ramon Pallàs-Areny.
''Direct sensor-to-microcontroller interface circuits, design and characterisation''. {\it Macombo, Barcelona}. (2005)

%%  Exemple de d'article en revista
%\bibitem{prova2} 
%Cognoms-autor, Inicial-nom.
%``Títol de l'article''. {\it Títol de la revista}.
%{\bf volum}(numero),
%100--110. (Any publicació) 

\end{thebibliography}

%%%%%%%%%%%%%%%%%%%%%%%%%%%%%%%%%%%%%%%%%%%%%%%%%%%%%%%%%%%%%%%%%%%%%%%%%%
%%%%%%                           APENDIXS                         %%%%%%%%
%%%%%%%%%%%%%%%%%%%%%%%%%%%%%%%%%%%%%%%%%%%%%%%%%%%%%%%%%%%%%%%%%%%%%%%%%%
\pagestyle{empty}  % no tocar

%% Descomentar una de les dues línies següents, en funció de:
%%  a) els apendixs s'encuadernaran apart (amb portada) 
%%  b) els apendixs s'enquadernen amb el mateix projecte (sense portada). 
%% Recordeu que si tot el document (amb apèndixs) excedeix les 100 pagines 
%% s'ha d'enquadernar a part
%\appendix\ambportada
\appendix\senseportada


%%%%%%%%%%%%%%%%%%%%%%%%%%%%%%%%%%%%%%%%%%%%%%%%%%%%%%%%%%%%%%%%%%%%%%%%%%
%%%%%% INCLOURE A PARTIR D'AQUI TOTS ELS CAPÍTOLS DELS APENDIXS   %%%%%%%%
%%%%%%%%%%%%%%%%%%%%%%%%%%%%%%%%%%%%%%%%%%%%%%%%%%%%%%%%%%%%%%%%%%%%%%%%%%

\chapter{Bytecodes}\label{Appx:Bytecodes}
Following list shows bytecodes supported by JVM designed for this project:
\begin{table}[!htb]
\centering
\begin{tabular}{|c|c|}
\hline 
Bytecode & Identifier \\
\hline 
NOP & 0x00 \\
\hline 
ICONST\_M1 & 0x02 \\
\hline 
ICONST\_0 & 0x03 \\
\hline 
ICONST\_1 & 0x04 \\
\hline 
ICONST\_2 & 0x05 \\
\hline 
ICONST\_3 & 0x06 \\
\hline 
ICONST\_4 & 0x07 \\
\hline 
ICONST\_5 & 0x08 \\
\hline 
BIPUSH & 0x10 \\
\hline 
SIPUSH & 0x11 \\
\hline 
ILOAD & 0x15 \\
\hline 
ILOAD\_0 & 0x1A \\
\hline 
ILOAD\_1 & 0x1B \\
\hline 
ILOAD\_2 & 0x1C \\
\hline 
ILOAD\_3 & 0x1D \\
\hline 
ISTORE\_0 & 0x3B \\
\hline 
ISTORE\_1 & 0x3C \\
\hline 
ISTORE\_2 & 0x3D \\
\hline 
ISTORE\_3 & 0x3E \\
\hline 
BASTORE & 0x54 \\
\hline 
DUP & 0x59 \\
\hline 
IADD & 0x60 \\
\hline 
ISUB & 0x64 \\
\hline 
IMUL & 0x68 \\
\hline 
IDIV & 0x6C \\
\hline 
IREM & 0x70 \\
\hline 
INEG & 0x74 \\
\hline 
ISHL & 0x78 \\
\hline 
ISHR & 0x7A \\
\hline 
IUSHR & 0x7C \\
\hline 
\end{tabular}
\quad
\begin{tabular}{|c|c|}
\hline 
Bytecode & Identifier \\
\hline 
IAND & 0x7E \\
\hline 
IOR &  0x80 \\
\hline 
IXOR & 0x82 \\
\hline 
IINC & 0x84 \\
\hline 
IFEQ & 0x99 \\
\hline 
IFNE & 0x9A \\
\hline 
IFLT & 0x9B \\
\hline 
IFGE & 0x9C \\
\hline 
IFGT & 0x9D \\
\hline 
IFLE & 0x9E \\
\hline 
IF\_ICMPEQ & 0x9F \\
\hline 
IF\_ICMPNE & 0xA0 \\
\hline 
IF\_ICMPLT & 0xA1 \\
\hline 
IF\_ICMPGE & 0xA2 \\
\hline 
IF\_ICMPGT & 0xA3 \\
\hline 
IF\_ICMPLE & 0xA4 \\
\hline 
GOTO & 0xA7 \\
\hline 
TABLESWITCH & 0xAA \\
\hline 
LOOKUPSWITCH & 0xAB \\
\hline 
IRETURN & 0xAC \\
\hline 
RETURN & 0xB1 \\
\hline 
GETSTATIC & 0xB2 \\
\hline 
PUTSTATIC & 0xB3 \\
\hline 
GETFIELD & 0xB4 \\
\hline 
PUTFIELD & 0xB5 \\
\hline 
INVOKEVIRTUAL & 0xB6 \\
\hline 
INVOKESPECIAL & 0xB7 \\
\hline 
INVOKESTATIC & 0xB8 \\
\hline 
NEWARRAY & 0xBC \\
\hline
\end{tabular}
\caption{Supported bytecodes}
\label{tab:Appx:Bytecodes}
\end{table}


\chapter{Code Structure}\label{Appx:CodeStructure}
OS native code has been organized as following structure:
\colorbox{yellow}{TODO}


\chapter{Native Application Code}\label{Appx:AppCodeNative}
Following C code shows the native application code used for direct interface circuit fo resistor sensors:
\medskip
\begin{lstlisting}[
caption={Native C application code},
captionpos=b,
label={cod:Appx:AppCodeNative:NativeCode},
language=C]
/*
 * File:   main.c
 * Author: Sergio Soria
 *
 * Created on 25 de enero de 2014, 17:05
 */

#include <xc.h>
#include <stdio.h>
#include "configuration.h"

#pragma config OSC      = HS
#pragma config IESO     = OFF
#pragma config FCMEN    = OFF
#pragma config WDT      = OFF
#pragma config BOREN    = OFF
#pragma config PWRT     = OFF
#pragma config MCLRE    = ON
#pragma config PBADEN   = OFF
#pragma config LVP      = OFF
#pragma config CPD      = OFF
#pragma config CP0      = OFF
#pragma config CP1      = OFF
#pragma config CP2      = OFF
#pragma config CP3      = OFF

#define _XTAL_FREQ 4000000
#define BAUDRATE 2400

// Booleans true and false values
#define TRUE    1
#define FALSE   0

// Byte value required to be received in order to start measures
#define RX_BYTE 0x01

// Application states
#define ST_MEASURE_NX   0
#define ST_MEASURE_NC1  1
#define ST_MEASURE_NC2  2

unsigned char TRISBValue = 0x00;
// Temporal array containing times measured
unsigned short g_Times[3];
// Temporal variable containing time measured
unsigned short g_Time = 0;

// If TRUE, measurements start. If FALSE, nothing happens
int g_Start     = FALSE;
// Current application state. Initially is measuring Nx
int g_State     = ST_MEASURE_NX;
// If TRUE, next time measurement can be done. If FALSE, not.
int g_Next      = FALSE;

void interrupt highISR(void)
{
#if (TIMER == TIMER1)
    // Stop Timer
    TMR1ON = 0;
    // Get Time
#ifdef CCP1
    g_Time = CCPR1;
#else // CCP1
    g_Time = READTIMER1();
#endif // (TIMER == TIMER1)

#elif (TIMER == TIMER2)
    // Stop Timer
    TMR2ON = 0;
    // Get Time
    g_Time = TMR2;
#endif // (TIMER == TIMER2)
    // Enable next time measurement
    g_Next = TRUE;

    // Clear interrupt
#ifdef CCP1
    CCP1IF = 0;
#else
    INT0IF = 0;
#endif // CCP1
}


void interrupt low_priority lowISR(void)
{
    if ((RCIF) && ((RCREG & RX_BYTE) == RX_BYTE)) {
        // Clear interrupt
        RCIF = 0;
        // If over run error, then reset the receiver
        if(OERR) {
            CREN = 0;
            CREN = 1;
        }
        // Indicates that measure can start
        g_Start = TRUE;
    }
}

#ifdef _STDIO_H_
void putch(char data)
{
    // Wait for previous transmission to finish.
    while (!TXIF) {
        continue;
    }

    // Write byte to send.
    TXREG = data;
}
#endif // _STDIO_H_

void main(void) {
    // Force to discharge capacitor through auxiliar Pin 5
    TRISB = 0x0F;
    PORTB = 0x00;

    __delay_ms(30);

    // Configure interrupts
    // --------------------
    //
    // Enable high and low priority interrupts int INT0 interrupt
    INTCON  = 0xD0;

#ifdef CCP1
    // CCP1 Interrupt enable bit
    CCP1IE = 1;
    // CCP1 interrupt as high priority
    CCP1IP = 1;
#else
    // Enable PORTB pull-ups and let RB0 interrupt on falling edge
    INTCON2 = 0x80;
#endif // CCP1

#ifdef CCP1
    CCP1CON = 0x05;
#else
    // If external Schmitt trigger is used in RB0, the edge interruption
    // should be swtiched to interrupt on rising edge
    INTEDG0 = 1;
#endif // CCP1

    // Enable interrupt priority
    IPEN = 1;

    // Configure Serial Port
    // ---------------------
    //
    // - RC6 = Asyncronous serial data ouput.
    // - RC7 = Syncronous serial receive data input.
    TRISC6 = 1;
    TRISC7 = 1;
    // - 8-bit Baud Rate
    BRG16 = 0;
    BRGH  = 0;
    SPBRG = ((_XTAL_FREQ / 64) / BAUDRATE) - 1;
    // - 8-bit transmission and reception
    TX9   = 0;
    // - Asynchronous mode
    SYNC  = 0;
    RX9   = 0;
    // Enable Serial port and receive mode
    SPEN  = 1;
    SREN  = 0;
    CREN  = 1;
    // Enable Rx interrupt and disable Tx interrupt
    RCIE  = 1;
    TXIE  = 0;

    // Set Rx interrupt as low level
    RCIP = 0;

    // Reset and enable transmitter
    TXEN  = 0;
    TXEN  = 1;

#ifdef CCP1
    // Configure CCP Timer
    // -------------------
    // - Timer 1 as CCP clock
    T3CON = 0x08;
    // RC2 as input
    TRISC2 = 1;
#endif //CCP1

#if (TIMER == TIMER1)
    // Configure Timer 1.
    // ------------------
    // - 16-bits
    // - 1:1 Prescale
    // - Internal clock
    //
    // Disables Timer1 overflow interrupt
    TMR1IE = 0;

    T1CON = 0x88;
    // Initially Timer 1 OFF
    TMR1ON = 0;

    // Clear counter values
    TMR1L = 0x00;
    TMR1H = 0x00;
#elif (TIMER == TIMER2)
    // Configure Timer 2.
    // ------------------
    // - 1:1 Postcale
    // - 1:16 Prescale
    //
    T2CON = 0x02;
    // Initially Timer 2 OFF
    TMR2ON = 0;
    // Disable the TMR2 to PR2 match interrupt
    TMR2IE = 0;
    // Clear counter values
    TMR2 = 0x00;
#endif // (TIMER == TIMER1)

    while (TRUE) {
        // Wait until receive byte to start
        while (g_Start) {
            // Charge capacitor throug auxiliar Pin 5
            TRISB = 0x0F;
            PORTB = 0x10;

            __delay_ms(30);

            // Next step is not goig to be executed until RB0
            // interrupt finishes
            g_Next = FALSE;

#if (TIMER == TIMER1)
            // Start Timer1
            WRITETIMER1(0x0000);
#elif (TIMER == TIMER2)
            // Start Timer 2
            TMR2 = 0x00;
            TMR2ON = 1;
#endif // TIMER

#ifdef CCP1
            CCPR1 = (unsigned short) 0;
#endif // CCP1

            switch (g_State) {
                case ST_MEASURE_NX:
                    // Discharge through Pin 2
                    TRISBValue = 0x1D;
                    break;
                case ST_MEASURE_NC1:
                    // Discharge through Pin 3
                    TRISBValue = 0x1B;
                    break;
                case ST_MEASURE_NC2:
                    // It is going to be the last measure
                    g_Start = FALSE;
                    // Discharge through Pin 4
                    TRISBValue = 0x17;
                    break;
            }

            TRISB = TRISBValue;

#if (TIMER == TIMER1)
            TMR1ON = 1;
#elif (TIMER == TIMER2)
            TMR2ON = 1;
#endif // (TIMER == TIMER1)

            // All values as logical '0'
            PORTB = 0x00;

            asm("L1: BTFSS _g_Next,0");
            asm("GOTO L1");

            // Store time
            g_Times[g_State] = g_Time;
            // Next step
            g_State++;

            // brief delay
            __delay_ms(20);
        }

        if (g_State > ST_MEASURE_NC2) {
            // Send values
            printf("%d;", g_Times[ST_MEASURE_NX]);
            printf("%d;", g_Times[ST_MEASURE_NC1]);
            printf("%d", g_Times[ST_MEASURE_NC2]);
            printf("\n");

            // Restart state
            g_State = ST_MEASURE_NX;
        }
    }
}

\end{lstlisting}
\medskip


\chapter{Java Application Code}\label{Appx:AppCodeJava}
Following Java code shows the application code used for direct interface circuit for resistor sensors using JVM for microcontrollers:
\medskip
\begin{lstlisting}[
caption={Java application code},
captionpos=b,
label={cod:Appx:AppCodeJava:JavaCode},
language=C]
package ujava.apps;

import java.mc.MicroApplication;
import java.mc.MicroApplicationListener;
import java.mc.ports.Port;
import java.mc.ports.PortConstants;
import java.mc.serialport.SerialPort;
import java.mc.serialport.SerialPortConstants;
import java.mc.timers.Timer;
import java.mc.timers.TimerConstants;

public class DirectSensor extends MicroApplication implements
  MicroApplicationListener {

    /**
     * Application state indicating next measure is Nx.
     */
    private final static byte STATE_MEASURE_NX = (byte) 0x00;

    /**
     * Application state indicating next measure is Nc1.
     */
    private final static byte STATE_MEASURE_NC1 = (byte) 0x01;

    /**
     * Application state indicating next measure is Nc2.
     */
    private final static byte STATE_MEASURE_NC2 = (byte) 0x02;

    /**
     * Java object representing microcontroller port resource.
     */
    private Port m_Port = null;

    /**
     * Java object representing microcontroller timer resource.
     */
    private Timer m_Timer = null;

    /**
     * Application variable indicating next measure to process.
     */
    private byte m_State = STATE_MEASURE_NX;

    /**
     * Auxiliar buffer
     */
    private byte[] m_AuxBuffer = {(byte) 0xFF, (byte) 0xFF};

    /**
     * Application starting method...
     */
    public void main() {
        // Store Microcontroller resource PORTB, that is
        // going to be modified during application to set
        // pins to '0' or '1'
        m_Port = Port.getPort(PortConstants.PORTB);
        // Store Microcontroller resource Timer1 used to
        // measure time spent to dischargue every capacitor
        // conected to PORTB pins
        m_Timer = Timer.getTimer(TimerConstants.TIMER1);

        // Dischargue all capacitors connected to PORTB
        m_Port.setIO((byte) 0x0F);
        m_Port.setPins((byte) 0x00);
        // Prudential time to dischargue capacitors
        MicroApplication.Sleep((short) 30);

        // Application initially is registered to receive
        // byte event in order to keep listening on Serial
        // Port pins
        MicroApplication.SetEvent(
          SerialPortConstants.EVENT_RECEIVED_BYTE);
        // Enable INT0 interrupt
        MicroApplication.SetEvent(
          PortConstants.EVENT_INTERRUPT_0);
    }

    /**
     * Method triggered when some event registered by
     * this application happens.
     *
     * @param event Event identifier.
     */
    public void onEvent(byte event) {
        // True if application is measuring, false if not
        boolean measuring = true;

        // Just for timer value accuracy, Timer is disabled
        // when some event is received.
        m_Timer.disable();

        // If some byte is received through serial port,
        // boolean m_StartMeasure is set to true in order to
        // indicate that measure can start
        if (SerialPortConstants.EVENT_RECEIVED_BYTE == event) {
            // Restart application state
            m_State = STATE_MEASURE_NX;
        }

        // If event INTO has been triggered, then it means
        // that one dischargue has been completed, then, store
        // result on global array Measures
        if (PortConstants.EVENT_INTERRUPT_0 == event) {
            // Read timer value
            short timerValue = m_Timer.read();

            // Copy timer value to auxiliar buffer
            m_AuxBuffer[(short) 0] = (byte) (timerValue >> 8);
            m_AuxBuffer[(short) 1] = (byte) timerValue;
            // Send value
            SerialPort.Send(m_AuxBuffer, (short) 0, (short) 2);

            // If it is last state, then send Measures result
            // and restart m_State variable in order to start
            // again if new byte is received
            if (m_State == STATE_MEASURE_NC2) {
                // Send end of line
                m_AuxBuffer[(short) 0] = (byte) '\n';
                // Restart application state
                m_State = STATE_MEASURE_NX;
                // Measuring finished
                measuring = false;
            } else {
                m_AuxBuffer[(short) 0] = (byte) ';';
            }
            
            // Send end of line or separator
            SerialPort.Send(m_AuxBuffer, (short) 0, (short) 1);

            // Next state
            m_State++;
        }

        // Application is measuring timings...
        if (measuring) {
            // Dischargue all capacitors connected to PORTB
            m_Port.setIO((byte) 0x0F);
            m_Port.setPins((byte) 0x10);

            // Prudential time to dischargue capacitors
            MicroApplication.Sleep((short) 30);

            // Clear Timer
            m_Timer.clear();

            // This variable will contain the configuration of
            // PORTB pins depending on current statre measure
            // to do. The configuration corresponding to pin
            // direction
            byte ioPinsConfiguration = (byte) 0x00;

            // Set PORTB pins direction configuration
            if (STATE_MEASURE_NX == m_State) {
                ioPinsConfiguration = (byte) 0x1D;
            } else
            if (STATE_MEASURE_NC1 == m_State) {
                ioPinsConfiguration = (byte) 0x1B;
            } else
            if (STATE_MEASURE_NC2 == m_State) {
                ioPinsConfiguration = (byte) 0x17;
            }

            // Set PORTB directions configuration
            m_Port.setIO(ioPinsConfiguration);
            // Start Timer
            m_Timer.enable();
            // PORTB pin values (no directions)
            m_Port.setPins((byte) 0x00);
        }
        
    }

}
\end{lstlisting}
\medskip


\end{document}
